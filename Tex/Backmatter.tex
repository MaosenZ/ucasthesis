%\chapter{作者简历及攻读学位期间发表的学术论文与研究成果}
\chapter{攻读学位期间发表的学术论文与会议报告}
%\textbf{本科生无需此部分}。

%\section*{作者简历}

%\subsection*{casthesis作者}

%吴凌云,福建省屏南县人,中国科学院数学与系统科学研究院博士研究生。

%\subsection*{ucasthesis作者}

%莫晃锐,湖南省湘潭县人,中国科学院力学研究所硕士研究生。

\section*{已发表(或正式接受)的学术论文:}

%[1] ucasthesis: A LaTeX Thesis Template for the University of Chinese Academy of Sciences, 2014.
[1] 主要作者(paper contact): Search for Higgs boson pair production in the $WW^*WW^*$ decay channel using ATLAS data recorded at $\sqrt{s}$= 13 TeV (CERN-EP-2018-227).
 
[2] 主要作者: LHC search of new Higgs boson via resonant di-Higgs production with decays into 4W (J. High Energ. Phys. (2018) 2018: 90). 

[3] 参与者: Search for Higgs boson pair production in the final state of $\gamma\gamma WW^*$($\rightarrow\ell\nu jj$) using 13.3 fb$^{-1}$ of $pp$ collision data recorded at 13 TeV with the ATLAS detector (ATLAS-CONF-2016-071).

[4] 参与者: Search for Higgs boson pair production in the $\gamma\gamma WW^*$ channel using $pp$ collision data recorded at $\sqrt{s}$=13 TeV with the ATLAS detector (Eur. Phys. J. C (2018) 78: 1007).

[5] 参与者: Expected Performance of the ATLAS Inner Tracker at the High-Luminosity LHC (ATL-PHYS-PUB-2016-025).

\section*{国际会议海报与报告}
[1] 报告: "Search for di-Higgs decaying to $WW^*WW^*$ with the final state of the same sign leptons plus 4 jets at LHC" at <<International Symposium on Higgs Boson and Beyond Standard Model Physics>>, 2016.

[2] 海报: "The production of additional bosons and the impact on Higgs boson physics" at <<The fifth Annual Large Hadron Collider Physics conference>>, 2017.

[3] 海报: "Search for di-Higgs production with the ATLAS detector" at  <<21st Particles and Nuclei International Conference>>, 2017.

[4] 报告: "Search for Higgs boson pair production in the $\gamma\gamma WW^*$ channel using $pp$ collision data recorded at $\sqrt{s}$ = 13 TeV with the ATLAS detector" at <<Higgs Hunting 2018>>, 2018.

[5] 报告: "The analysis of $HH\rightarrow WW^*\gamma\gamma$ with the ATLAS detector" at <<Double Higgs Production at Colliders Workshop>>, 2018.
%\section*{申请或已获得的专利:

%(无专利时此项不必列出)

%\section*{参加的研究项目及获奖情况:}

%可以随意添加新的条目或是结构。

\chapter[致谢]{致\quad 谢}\chaptermark{致\quad 谢}% syntax: \chapter[目录]{标题}\chaptermark{页眉}
\thispagestyle{noheaderstyle}% 如果需要移除当前页的页眉
%\pagestyle{noheaderstyle}% 如果需要移除整章的页眉

%感激casthesis作者吴凌云学长,gbt7714-bibtex-style
%开发者zepinglee,和ctex众多开发者们。若没有他们的辛勤付出和非凡工作,\LaTeX{}菜鸟的我是无法完成此国科大学位论文\LaTeX{}模板ucasthesis的。在\LaTeX{}中的一点一滴的成长源于开源社区的众多优秀资料和教程,在此对所有\LaTeX{}社区的贡献者表示感谢!
%
%ucasthesis国科大学位论文\LaTeX{}模板的最终成型离不开以霍明虹老师和丁云云老师为代表的国科大学位办公室老师们制定的官方指导文件和众多ucasthesis用户的热心测试和耐心反馈,在此对他们的认真付出表示感谢。特别对国科大的赵永明同学的众多有效反馈意见和建议表示感谢,对国科大本科部的陆晴老师和本科部学位办的丁云云老师的细致审核和建议表示感谢。谢谢大家的共同努力和支持,让ucasthesis为国科大学子使用\LaTeX{}撰写学位论文提供便利和高效这一目标成为可能。

\cleardoublepage[plain]% 让文档总是结束于偶数页,可根据需要设定页眉页脚样式,如 [noheaderstyle]

