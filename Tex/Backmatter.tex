%\chapter{作者简历及攻读学位期间发表的学术论文与研究成果}
\chapter{攻读学位期间发表的学术论文与会议报告}
%\textbf{本科生无需此部分}。

\section*{作者简历}
2010年9月—2014年6月,在四川大学物理科学与技术学院获得学士学位。

2014年9月—2019年6月,在中国科学院高能物理研究所攻读博士学位。
%\subsection*{casthesis作者}

%吴凌云,福建省屏南县人,中国科学院数学与系统科学研究院博士研究生。

%\subsection*{ucasthesis作者}

%莫晃锐,湖南省湘潭县人,中国科学院力学研究所硕士研究生。

\section*{已发表(或正式接受)的学术论文:}

%[1] ucasthesis: A LaTeX Thesis Template for the University of Chinese Academy of Sciences, 2014.
[1] 主要作者(paper contact): Search for Higgs boson pair production in the $WW^*WW^*$ decay channel using ATLAS data recorded at $\sqrt{s}$= 13 TeV (CERN-EP-2018-227).
 
[2] 主要作者: LHC search of new Higgs boson via resonant di-Higgs production with decays into 4W (J. High Energ. Phys. (2018) 2018: 90). 

[3] 参与者: Search for Higgs boson pair production in the final state of $\gamma\gamma WW^*$($\rightarrow\ell\nu jj$) using 13.3 fb$^{-1}$ of $pp$ collision data recorded at 13 TeV with the ATLAS detector (ATLAS-CONF-2016-071).

[4] 参与者: Search for Higgs boson pair production in the $\gamma\gamma WW^*$ channel using $pp$ collision data recorded at $\sqrt{s}$=13 TeV with the ATLAS detector (Eur. Phys. J. C (2018) 78: 1007).

[5] 参与者: Expected Performance of the ATLAS Inner Tracker at the High-Luminosity LHC (ATL-PHYS-PUB-2016-025).

\section*{国际会议海报与报告}
[1] 报告: "Search for di-Higgs decaying to $WW^*WW^*$ with the final state of the same sign leptons plus 4 jets at LHC" at <<International Symposium on Higgs Boson and Beyond Standard Model Physics>>, 2016.

[2] 海报: "The production of additional bosons and the impact on Higgs boson physics" at <<The fifth Annual Large Hadron Collider Physics conference>>, 2017.

[3] 海报: "Search for di-Higgs production with the ATLAS detector" at  <<21st Particles and Nuclei International Conference>>, 2017.

[4] 报告: "Search for Higgs boson pair production in the $\gamma\gamma WW^*$ channel using $pp$ collision data recorded at $\sqrt{s}$ = 13 TeV with the ATLAS detector" at <<Higgs Hunting 2018>>, 2018.

[5] 报告: "The analysis of $HH\rightarrow WW^*\gamma\gamma$ with the ATLAS detector" at <<Double Higgs Production at Colliders Workshop>>, 2018.
%\section*{申请或已获得的专利:

%(无专利时此项不必列出)

%\section*{参加的研究项目及获奖情况:}

%可以随意添加新的条目或是结构。

\chapter[致谢]{致\quad 谢}\chaptermark{致\quad 谢}% syntax: \chapter[目录]{标题}\chaptermark{页眉}
\thispagestyle{noheaderstyle}% 如果需要移除当前页的页眉
%\pagestyle{noheaderstyle}% 如果需要移除整章的页眉

首先感谢方亚泉老师的悉心指导与支持,为我提供了良好的学术渠道,也给予了我极大的学术自由。
感谢小组师兄及同事:杜春,孙小虎,张慧君,李奇,章宇及张凯栗。
杜春师兄指导了我的本科毕业论文,是他为我打开了LHC唯象物理研究的大门;
孙小虎师兄给予了我统计学和MC产生方面的指导,尤其是步入ATLAS前期,我从他那里得到了许多技术方面的帮助;
张慧君师兄为$WW^*\gamma\gamma$分析打下了坚实的第一步,高兴于他讲述的各种有趣八卦;
感谢李奇师兄给予的关于统计工具和各种高能所行政流程的帮助;
章宇思维清晰,与他讨论,总是受益匪浅;
张凯栗不辞麻烦,当我在美国期间,帮我处理了许多行政手续,并向他请教了许多关于CEPC的问题。

感谢ATLAS Upgrade Tracking分析组的会议召集人和同事,他们给予了作为新手的我极大的宽容与帮助,
也感激他们推荐我作关于Upgrade tracking performance的国际会议报告;
特别是Heather作为我Qualification task的supervisor,前期给予了我许多关于tracking的指导,她的sharp也给我留下了深刻印象;
还想特别感谢Tim和Shih-Chieh对我工作的支持与理解。

感谢4W分析相关的老师、会议召集人及同事。
感谢Biagio提供的diHiggs NLO的LHE文件;
感谢Arnaud给4W提出的许多建设性意见;
感谢李亮老师,李兴国,Jason,Yesenia,没有他们,4W分析不能圆满完成,尤其是李兴国和Jason极大的推进了论文写作及提交。
感谢4W分析的Editorial board:Magda, James, Emmanuel,特别是Magda多次阅读我的supporting note,并给出了许多有用意见。
在随后参与的tth分析中,感谢Tamara, Richard及Ximo对工作的合理组织及安排,感谢Ntuple组的辛勤工作。

有幸在伯克利实验室交流一年半,收获颇丰。
首先要感谢姚为民老师,他可算我的第二导师,从Qualification到$t\bar{t}h$分析,均从他那里得到了有益的指导与建议,尤其是在伯克利期间,他还给予了我无微不至的照顾与指导,在此特别感谢。
还要感谢我的硬件导师Sandra,她引领我进入了Strips模块的组装与测试工作;
感谢Charilou,是她领我熟悉了整个工作流程,并告知我许多关于Strips的有用信息;
感谢Luise,她的到来,极大地推进了整个工作,也感谢她对我的指导与帮助,她的sharp和组织能力也给我留下了深刻印象;
感谢Timon与Karol提供Power board和电子学方面的帮助;
感谢Rhonda和Phat的wire-bonding, Tom和Willy的tooling及gluing, Ken的设备保障支持,Earl的尺度测量支持,没有他们的帮助与支持,我们不能制造一个模块。
美国内部Strips工作交流紧密,在此特别指出,Santa Cruz的Vitaliy与Matthew前期对我们的模块组装及测试工作助力极大,布鲁克海文的Stefania成功帮助我解决第一块模块测试时的debug工作。
另外,感谢ATLAS组领导Ian, 感谢Quanita每月为我提交补助表格,感谢Maurice提供的办公室座位(虽未在Pixel组工作)。
在伯克利实验室的工作是非常愉快的,感谢ATLAS组的所有成员,我从他们身上真正看到了务实、严谨及不骄不躁的精神,在硬件工作中锻炼了我的交流与组织能力;
我也重新审视到底什么是研究,研究可大可小,但不应有高低。

当然,还要感谢我的培养单位,高能所,它为我提供了从事高能物理实验的平台;
%感谢它每月为我报销大部分药费,减轻了许多经济负担;
感谢高能所的ATLAS组,尤其感谢庄胥爱老师每周兢兢业业的会议组织工作;
感谢Joao对大家工作的评论,往往能够发现我从未注意到的漏洞;
感谢我在高能所认识的同学及朋友们,与他们交流工作,与他们一起玩耍,总是非常愉快的。

感谢我的家人,感恩他们二十多年来对我的养育,他们的期望,一直是我的前进动力之一。
感谢我的女朋友对我的信任、肯定与包容,尤其是论文写作期间,忍让我对她的忽视,感激不尽。
我想,我还有许许多多人应当感谢,可是并不能一一列出,所以最后在此一并感谢所有对我的成长有过帮助的人!

\cleardoublepage[plain]% 让文档总是结束于偶数页,可根据需要设定页眉页脚样式,如 [noheaderstyle]

