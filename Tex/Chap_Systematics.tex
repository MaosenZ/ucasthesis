\section{系统误差}\label{sec:4w_systematics}

\subsection{对撞亮度}
2015年和2016年的联合亮度的不确定性为2.1\%,将应用到信号样本和promptSS本底上。亮度误差估计利用与文献~\cite{DAPR-2013-01}类似的方法得出,
来自于2015年8月和2016年5月进行的xy束流分离扫描对亮度的初步校准结果。

\subsection{信号模型}
\begin{itemize}
 \item 标准模型希格斯对的信号样本的理论误差依据推荐~\cite{LHCdiHiggsXsec}是\\ 
 $^{+4.3}_{-6.0}\text{(scale)} ^{+5.0}_{-5.0}\text{(Th.)}  ^{+2.1}_{-2.1}\text{(PDF)} ^{+2.3}_{-2.3} (\alpha_S)$\%。
  \item 共振态希格斯对信号样本利用CT10 PDF(具有26个相互独立参数)产生,其来源于PDF的误差通过\texttt{LHAPDF6}~\cite{Buckley:2014ana}估计,
  每个产生的事例会根据PDF参数变动被重新赋予权重:\\
  \begin{equation}
w_{i}=\frac{x_1f_{1i}(x_1;Q) x_2f_{2i}(x_2;Q)}{x_1f_{10}(x_1;Q) x_2f_{20}(x_2;Q} (i=1,2...,52),
  \end{equation}
  其中``1''和``2''表示硬散射过程中的两个入射部分子,``0''是PDF的最佳拟合值,即基准值,而$i$对应PDF参数的一次向上或向下变动。
  每项PDF参数变动之后的事例数与使用基准值的事例数差别作为单项误差,
  而总的PDF误差取它们的平方和。对于$SS$样本,PDF误差则是NNPDF23中的100项参数变动带来的误差的平方和,结果大约为6\%。
   \item QCD重整化和因子化参数同时或者单独加倍(或者减半),然后计算相应变动之后的事例数,最终把与基准值差别最大的一次作为系统误差,对于$hh$($SS$)可达到10\%(4\%)。
   \item 部分子簇射产生子的选择可影响信号的接收效率。对于$hh$信号,使用\PYTHIAV{8} 作为部分子簇射产生子的信号样本被产生,而后比较其与使用\Herwigpp 的基准信号
   接收效率的差别,此差别将作为部分子簇射产生模型的系统误差,随着$m_X$的增大,其值从40\%到10\%;而对于$SS$,\PYTHIAV{8} 内部的所有部分子簇射模型参数变动带来的差别
   的平方和将作为此项误差,其值大概为10\%。
\end{itemize}
所有的信号样本的理论误差总结在表~\ref{tab:sig_theory_hh}和表~\ref{tab:sig_theory_SS}中。
\begin{table}[!htp]
\tiny
\centering
 \begin{tabular}{c|cccccccccccccccccccccc}
\hline
 $hh$ &\multicolumn{3}{c}{SM Higgs pair} &\multicolumn{3}{c}{$m_X$=260 GeV} &\multicolumn{3}{c}{$m_X$=300 GeV} &\multicolumn{3}{c}{$m_X$=400 GeV}  &\multicolumn{3}{c}{$m_X$=500 GeV} \\
\hline
     &$ee$  &$\mu\mu$  &$e\mu$  &$ee$  &$\mu\mu$  &$e\mu$  &$ee$  &$\mu\mu$  &$e\mu$ &$ee$  &$\mu\mu$  &$e\mu$ &$ee$  &$\mu\mu$  &$e\mu$ \\
PDF &4.01 &4.07 &4.09
    &3.88 &3.80 &3.92 
    &3.87 &3.78 &3.86 
    &3.85 &3.75 &3.83 
    &3.98 &3.94 &3.95 \\ 
PS  &13.24&18.18&10.00
    &26.58&42.65&19.28
    &30.72&31.65&18.96
    &22.02&24.48&33.14
    &1.64&16.59&12.76
\\
Scale &1.39 &1.15 &6.96
      &5.78 &3.97 &0.06
      &0.13 &4.85 &0.02
      &9.86 &0.09 &3.07
      &3.91 &0.98 &1.05 \\
\hline
\end{tabular}
\caption{The theoretical uncertainties on $X\rightarrow hh$ production.}
\label{tab:sig_theory_hh}
\end{table}

\begin{table}[!htp]
\centering
\scalebox{0.5}{
 \begin{tabular}{c|cccccccccccccccccccccccccccccc}
\hline
 $hh$ &\multicolumn{3}{c}{$m_X$=280 GeV, $m_S$=135 GeV} &\multicolumn{3}{c}{$m_X$=300 GeV, $m_S$=135 GeV} &\multicolumn{3}{c}{$m_X$=320 GeV, $m_S$=135 GeV}  &\multicolumn{3}{c}{$m_X$=340 GeV, $m_S$=135 GeV} &\multicolumn{3}{c}{$m_X$=340 GeV, $m_S$=145 GeV}
&\multicolumn{3}{c}{$m_X$=340 GeV, $m_S$=155 GeV} &\multicolumn{3}{c}{$m_X$=340 GeV, $m_S$=165 GeV}\\
\hline
    &$ee$  &$\mu\mu$  &$e\mu$  &$ee$  &$\mu\mu$  &$e\mu$ &$ee$  &$\mu\mu$  &$e\mu$ &$ee$  &$\mu\mu$  &$e\mu$ 
  &$ee$  &$\mu\mu$  &$e\mu$  &$ee$  &$\mu\mu$  &$e\mu$ &$ee$  &$\mu\mu$  &$e\mu$ \\
PDF &5.75 &5.79 &5.80
    &5.86 &5.87 &5.86 
    &5.98 &6.06 &6.06 
    &6.07 &6.37 &6.22 
    &6.06 &6.14 &6.10 
    &6.05 &6.22 &6.25 
    &6.10 &6.18 &6.14 \\ 
PS  &4.97&6.50&6.95
    &5.97&8.12&2.86
    &3.90&10.69&7.44
    &8.65&6.16&5.83
    &10.15&6.24&6.51
    &7.41&7.32&9.18
    &9.29&8.74&7.85
\\
Scale &0.55 &2.22 &3.90
      &0.55 &2.22 &3.90
      &0.55 &2.22 &3.90
      &0.36 &3.32 &1.16
      &0.36 &3.32 &1.16
      &0.36 &3.32 &1.16
      &0.36 &3.32 &1.16 \\
\hline
\end{tabular}}
\caption{The theoretical uncertainties on $X\rightarrow SS$ production.}
\label{tab:sig_theory_SS}
\end{table}


\subsection{data-driven本底}
QmisID和fakes本底的误差如章节~\ref{sec:bkg_estimation}所示,值得指出的是,QmisID的系统误差对其本身和fakes上具有相反的影响。

\subsection{MC本底预期截面}
promptSS和$V\gamma$本底利用MC估计,其中主要部分是$WZ$(70\%),根据三轻子分析道的估计,$WZ$的截面误差为25\%。$W^{\pm}W^{\pm}$本底在promptSS占据大约10\%,其误差
假设为50\%。剩下的$V\gamma$,$tV$,$t\bar{t}V$和$t\bar{t}H$的预期截面误差均假设为50\%,为了支持50\%是一个足够保守的估计,可参见标准模型截面测量实验,$tZ$和$ttW$($ttZ$)的
截面测量误差分别为15\%~\cite{xsection_WZ}和53.3\%(33\%)~\cite{xsection_ttV}。

\subsection{实验相关}
实验相关误差主要有以下几项:
\begin{itemize}
  \item 影响信号运动学分布的系统误差
   \begin{itemize}
     \item 电子沉积能量测量和分辨率
     \item 受低动量径迹影响的\met 重建误差
     \item Jet能量测量及分辨误差
   \end{itemize}
   \item 末态粒子重建及选择带来的效率修正误差
    \begin{itemize}
      \item 轻子重建,鉴别以及孤立效率
      \item Pile-up reweighting
      \item JVT event weight
      \item \btag 效率
    \end{itemize}
\end{itemize}
信号MC,promptSS和$V\gamma$同时考虑了这些误差,但在实际操作中,每个分析道每个信号区各自舍弃掉整体影响低于0.5\%的系统误差。表~\ref{tab:summary_syst_ee_nonres}总结了在$hh$搜寻中$e\mu$分析道的系统误差大小。
\begin{table}[h]
\begin{center}
%\scriptsize
\begin{tabular}{c|ccc|cccc}
\hline
Uncertainty source  &Non-resonant $hh$  &PromptSS &$V\gamma$ &Fakes &QmisID\\
\hline
Luminosity &$\pm$2.1 &$\pm$2.1 &$\pm$2.1 &$\mp$2.1  &\multirow{ 16}{*}{}\\
\hline
PDF  &$\pm$2.1  &   & &\multirow{ 3}{*}{}\\
Scale &+4.3/-6.0 & & &\\
Top mass &$\pm$5.0 & & &\\
$\alpha_S$ &$\pm$2.3 & & &\\
\hline
$WZ$ cross-section  & &$\pm$12.5 & &$\mp$19.8\\
$ssWW$ cross-section & &$\pm$8.3 & &$\mp$6.1\\
$ttV$ cross-section & &$\pm$8.1 & &\\
$tV$ cross-section & &$\pm$1.2 &  &\\
$ttH$ cross-section & &$\pm$1.9 & & \\
$V\gamma$ cross-section & & &$\pm$50 &\\
\hline
Pile-up reweighting &$\pm$3.63 &$\pm$2.24 &$\pm$20.48 &\multirow{ 5}{*}{}\\
b-tagging &$\pm$2.63 &$\pm$2.8 & &\\
JVT &$\pm$0.78 &$\pm$0.61  &$\pm$0.6 &\\
lepton ID &$\pm$1.1 &$\pm$1.2 &$\pm$1.0 &\\
\cline{1-4}
JES/JER &$\pm$4.0 & $\pm$14.7 &$\pm$98 &\\
\cline{1-4}
MET &$\pm$0.8 &$\pm$1.24 & &\\
\hline
QmisID &  &  &  &$\mp$16.1 &$\pm$30\\
\hline
\end{tabular}
\caption{对应non-resonant $hh$ $e\mu$搜寻的系统误差(\%)总结。}
%\caption{The summary of systematic uncertainty for the search of non-resonant $hh$ in $e\mu$ channel. All numbers are in \%. It should be noted that the non-closure, sample composition and stat. uncertainties are not shown for fakes.}
\label{tab:summary_syst_ee_nonres}
\end{center}
\end{table}
