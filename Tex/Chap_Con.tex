%\chapter{结论}\label{chap:conclusions}
%希格斯粒子作为新物理的窗口(portal),对它的精确测量可以指引前进道路。
%顶夸克汤川耦合测量作为希格斯粒子性质测量的重要一环,可以帮助我们
本文论述了通过多轻子衰变道寻找标准模型(SM)希格斯粒子关联顶夸克对产生模式,使用2015年到2017年ATLAS探测器收集的积分亮
度为$\int \mathcal{L}dt~=~80~\text{fb}^{-1}$的质子-质子对撞数据。
主要研究末态是由一个轻子和两个强子化衰变的$\tau$组成的分析道\ltwotau ,大部分信号来自$t\bar{t}h(\rightarrow \tau\tau)$,
主要本底是来源于$t\bar{t}$的假$\tau_{\text{had}}$。利用相同电荷两$\tau_{\text{had}}$数据(SS data)可以很好地模拟假$\tau_{\text{had}}$本底,并且在控制区域得到证实,
其相对偏差和SS data本身统计误差将作为假$\tau_{\text{had}}$本底的系统误差。
其余本底主要是$t\bar{t}Z$,利用MC估计。
因为$t\bar{t}$ MC与模拟假$\tau_{\text{had}}$本底的SS data的形状差异较小,可用于多变量方法训练以进行信号优化。
事例根据BDTG划分为三个区域,同时拟合这三个区域给出信号强度为$\mu_{t\bar{t}h}$=$\sigma/\sigma_{\text{SM}}$=$1.00^{+1.22}_{-1.02}$,相当于
$0.96\sigma$的信号显著性。

本文随后论述了通过多轻子道寻找一对衰变到四个$W$玻色子的中性标量玻色子,使用2015年到2016年ATLAS探测器收集的积分亮度为$\int \mathcal{L}dt=36.1~\text{fb}^{-1}$的质子-质子对撞数据。
该搜寻包括三种信号模型:SM非共振态产生($hh$),共振态产生($X\rightarrow hh$)和有两个重质量类希格斯粒子的共振态产生($X\rightarrow SS$)。
主要研究了相同电荷双轻子衰变道,根据轻子味道分为$ee$, $\mu\mu$以及$e\mu$子类。$ee$和$e\mu$子类本底主要有可贡献双轻子末态过程(promptSS),光子转换电子($V\gamma$),
假轻子(fakes)以及电子电荷误判(QmisID),而$V\gamma$和QmisID本底在$\mu\mu$子类基本可忽略。
promptSS和$V\gamma$使用MC估计。
对于QmisID本底,首先在$Z$控制区利用似然函数方法得到电子电荷误判率,随后应用到信号区相反电荷双轻子事例而得到。
Fakes使用所谓的fake factor method估计,该方法假设fakes中通过tight轻子筛选的事例与通过anti-tight轻子的事例的比例在控制区和信号区一致,
计算出控制区的比例(fake factor)则可推出信号区的fakes。Fakes本底的估计误差对结果影响最大。
随后利用多变量方法实现基于运动学变量的最佳筛选条件。
因为总体上通过筛选之后的事例数不高,所以采用单bin拟合(event counting experiment)。
最后联合其他两个分析道,三轻子(3L)和四轻子(4L),结果显示无明显超出标准模型迹象。
所以在95\%置信度下设置截面上限,
SM非共振态产生截面观测(期望)上限值为160倍(120倍)预期。
$X\rightarrow hh$在260 GeV $\leq m_X\leq$ 500 GeV的质量区间内的产生截面观测(期望)上限值为9.3 (10) pb到2.8 (2.6) pb。
$X\rightarrow SS$在280 GeV $\leq m_X\leq$ 340 GeV, 135 GeV $\leq m_S\leq$ 165 GeV的质量区间内产生截面观测(期望)上
限值为2.5 (2.5) pb到0.16 (0.17) pb。

对于以上物理分析结果,大致可以通过以下几方面得到提升:
\begin{itemize}
 \item 更多数据:利用全部\RunTwo 数据,联合希格斯粒子的其他衰变道分析($h\rightarrow \gamma\gamma,~h\rightarrow b\bar{b}$),将有望达到$5\sigma$信号显著性。
随着在Run~3 以及Phase-II 升级之后的HL-LHC累积更多数据,也许还可通过比$t\bar{t}h$产生截面更小的单顶夸克产生($th$或$tWh$)限制顶夸克汤川耦合。
对于$hh$分析,类似地,更多的数据必将压低$hh$的产生截面(如果没有信号)从而更加严格地验证标准模型或者发现新物理。
 \item 分析技术优化:比如假$\tau_{\text{had}}$本底的估计,可以增大SS data统计量以减小系统误差;2LSS分析中寻找更优的轻子筛选条件以压低fakes和减小系统误差。
 \item ATLAS实验改进:从底层到数据分析的每一环的改进也会有利于最终的物理结果呈现,比如触发,寻迹,
粒子重建及鉴别等等,特别是不断发展的机器学习方法对ATLAS实验有极大的裨益。
\end{itemize}

最后值得一提的是,$t\bar{t}h$对$hhh$顶点修正最敏感\cite{Maltoni:2017ims},所以通过单希格斯粒子产生模式可间接限制$\kappa_{\lambda}$,未来还可期望联合拟合单希格斯粒子与希格斯粒子对实验结果以
提高显著性。总而言之,$t\bar{t}h$和$hh$寻找仅是一个开端,希格斯物理是LHC未来的研究重点,因为希格斯粒子是一个窗口,通过它,可以验证标准模型,也可指明新物理方向。
