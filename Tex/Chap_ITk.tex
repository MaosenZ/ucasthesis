\section{ATLAS Phase-II升级}
LHC会在2020年之后进行一个主要升级,称为High Luminosity LHC (HL-LHC),届时LHC将在14 TeV能量下运行,瞬时亮度将达到原始设计目标的5倍,即$7.5\times10^{34}~\text{cm}^{-2}s^{-1}$,
LHC的计划运行和升级时间表可见图\ref{fig:LHC_timeline}。
HL-LHC的成功运行依赖于超导磁铁等新技术的运用,具体可见HL-LHC技术设计报告\cite{Apollinari:2284929}。
\begin{figure}[h]
\centering
 \includegraphics[width=0.85\textwidth]{fig/HL-LHC-PLAN.png}
 \caption{LHC运行和升级时间表\cite{Apollinari:2284929}。}
 \label{fig:LHC_timeline}
\end{figure}
为了承受HL-LHC的高\pileup 和高辐照压力,ATLAS探测器将进行一个全面升级。内部径迹探测器将全部替换成由硅组成的Inner Tracker (ITk),
ITk分为靠近束流的像素探测器(Pixel)\cite{Collaboration:2285585}和扩展到高半径的硅微条探测器(Strips)\cite{Collaboration:2017mtb},Pixel桶部区有五层,Strips则有四层,
一系列环形探测器也会添加到前向区使得寻迹区域扩展到$\abseta<4.0$。
Liquid Argon (LAr)量能器\cite{Collaboration:2285582}将会有全新的前端和读出电子学器件,其电子学架构设计在40 MHz输出全粒度数字信号(full-granularity digitized signals)。
Tile量能器\cite{Collaboration:2285583}会使用新的前端和读出电子学器件,电源和光链路接口板(optical link interface boards)。
$\mu$子探测器\cite{Collaboration:2285580}的一大部分前端,在和不在探测器(on- and off-detector)的读出和触发电子学设备将会被替换,额外的$\mu$子室也会安装以保持$\mu$子的鉴别和重建性能,
另外目前正在研究扩展到$\abseta<4.0$的可能性。
全新的触发和数据接收系统\cite{Collaboration:2285584}也会应用在升级ATLAS上。
考虑到HL-LHC的高pileup,一个新的探测器High-Granularity Timing Detector (HGTD)\cite{Collaboration:2623663}会安装在LAr量能器之前,覆盖$2.4<\abseta<4.0$区域,它可以精确测量带电粒子的时间分辨。
本章将关注硅微条探测器模块和ITk的预期径迹重建性能研究。

\subsection{Strips模块组装及测试}
ATLAS Phase-II升级之后的硅微条探测器传感器覆盖165 m$^2$,桶部区有四层,而端部磁盘区有六层,总共需要建造18,000个基本探测模块。一个完整模块由一个
硅传感器,两块或一块包含读出芯片(ABC130Star chip)以及控制芯片(HCCStar)的支撑电路板(hybrid)和电源板(Power Board)组成,具体可见图\ref{fig:strips_module}.
虽然Strips硅传感器的大小和形状由模块所在位置决定(见图\ref{fig:strip_sensor_overview}),但是其整体设计和架构是一致的。读出芯片是二元设计(binary design),与每条微条(strip)通过
金属线相连,信号通过芯片初步处理之后传输到控制芯片,然后统一输出到读出系统,控制芯片同时负责传输控制命令,而电源板则是用于模块电源控制以及环境参数监控,各部分的具体
信息可见设计报告\cite{Collaboration:2017mtb}。本文将呈现整个模块的组装过程以及测试结果,所使用的读出芯片和控制芯片版本分别为ABC130和HCC,具体区别同样可见\cite{Collaboration:2017mtb}。

\begin{figure}[h]
\centering
 \includegraphics[width=0.85\textwidth]{fig/strips_module_3d.png}
 \caption{(short-strip) Strips模块。}
 \label{fig:strips_module}
\end{figure}

\begin{figure}[h]
\centering
 \includegraphics[width=0.85\textwidth]{fig/strips_sensor_overview.png}
 \caption{ITk Strips传感器种类\cite{Collaboration:2017mtb},Strips探测器桶部区内两层由短条(short-strip)传感器组成,外两层由长条(long-strip)组成,由于端部的扇形几何结构,其strip间距(pitch)随半径不同而不同。}
 \label{fig:strip_sensor_overview}
\end{figure}
模块的组装是探测器建造的重要一环,模块的质量决定了最终探测器的性能。
组装过程大致可以分为以下几步:
\begin{enumerate}
 \item 将读出芯片通过胶水粘合在支撑电路板的对应位置,而后连接读出芯片与支撑电路板(wire-bonding);
 \item 将芯片通过胶水粘合在传感器上,而后将每条硅微条与读出芯片相连;
 \item 最后粘合电源板到传感器上,最后进行wire-bonding。
\end{enumerate}
其实际对应过程可见图片\ref{fig:strip_moduel_assembly}\footnote{过程中的具体粘合方法,所使用的器械以及工艺等,一直在进行优化,不同组装地点也可能使用不同的方法。}。
组装的挑战在于各个部分的接触面大小,胶水厚度以及位置的精确控制,一般要求控制到$\mathcal{O}(10)~\mu\text{m}$水平,比如芯片与hybrid的胶水粘合面应当保证至少50\%的接触面,
以便传导芯片工作时产生的大量热量,而固化的胶水应当保持在120$\pm40~\mu\text{m}$厚度。
\begin{figure}[h]
\centering
 \includegraphics[width=0.85\textwidth]{fig/strips_module_assembly.png}
 \caption{ITk (short-strip)模块实际组装过程(省略wire-bonding过程),粘合芯片的胶水需要紫外线照射固化。}
 \label{fig:strip_moduel_assembly}
\end{figure}
组装完成的模块需要进行电子学测试,以保证模块工作性能达到预期。测试时,模块应当处于干燥且低温的环境中,其测试系统如图\ref{fig:strips_testing_setup}所示。
一个重要指标是测试ABC130芯片的输入噪声(input noise),它由芯片本身和与之相连的硅微条噪声决定。
其测试基本原理是阈值扫描,给定注入电荷,逐步提高阈值,效率慢慢变为0(效率50\%时对应的点为Vt50),拟合所得到的效率随阈值变化曲线,得到标准差($\sigma$);
在不同输入电荷情况下,拟合Vt50随电荷的变化曲线,得到斜率,即增益(gain)。$\sigma/\text{gain}$为输入噪声。
图\ref{fig:strips_testing_noise}展示几个模块粘合电源板之前和之后的输入噪声水平,基本在600电子与900电子之间(模块经受辐照之后,整体噪声会上升)。图\ref{fig:strips_bad_dist}则为测试模块的
工作不良硅微条的空间分布。总体而言,所示的组装模块均工作良好,达到预期。
\begin{figure}[h]
\centering
 \includegraphics[width=0.70\textwidth]{fig/strips_module_setup.png}
 \caption{Strips模块测试系统设置。}
 \label{fig:strips_testing_setup}
\end{figure}
\begin{figure}[h]
\centering
 \includegraphics[width=0.65\textwidth, angle=-90]{fig/LH_noise.pdf} \\
  \includegraphics[width=0.65\textwidth, angle=-90]{fig/RH_noise.pdf} 
 \caption{Strips模块输入噪声,上图对应左手Hybrid (LH),下图对应右手Hybrid (RH), 点线为粘合PB之前,'+'为粘合PB之后,每图底部为粘合PB之后与之前的噪声之比。
 总体上比例在1附近,但因为每次实际测试时,环境参数并不完全一致,所以不能够表明粘合PB之后噪声一定增加或者减少了。
下图底部的“鼓包”由电源板上的屏蔽盒(shieldb-box,内部有DC-DC converter)引起,本文写作期间正在研究解决方案。}
 \label{fig:strips_testing_noise}
\end{figure}
\begin{figure}[h]
\centering
\begin{subfigure}[b]{0.45\textwidth}
\centering
 \includegraphics[width=0.9\textwidth,angle=-90]{fig/Hists_Bad_wPB.pdf}
 \caption{}
\end{subfigure}
\begin{subfigure}[b]{0.45\textwidth}
\centering
 \includegraphics[width=0.9\textwidth,angle=-90]{fig/BadDistr_LBL-EL-08_wPB.pdf}
 \caption{}
\end{subfigure}
\caption{(a)为测试模块strip整体不良率;(b)为LBL-EL-08模块不良strip在传感器上的分布情况。}
\label{fig:strips_bad_dist}
\end{figure}

\subsection{ITk预期寻迹性能}
相比目前的内部径迹探测器,ITk具有更低的物质量\footnote{目前研究显示材料预算在TDR被低估,这些预期结果也许过于乐观,全新的估计正在进行。},其具有更高的寻迹效率和分辨。
径迹重建从种子(seed)开始,所谓的seed是三个空间点。
在ITk的前期研究中,只考虑了两种seed,全部由Pixel着火点组成的PPP,全部由Strips着火点组成的SSS。
图\ref{fig:ITk_seeds}展示seeds随\abseta 分布,整体而言,PPP比SSS数目高一个量级。
联合拟合PPP与SSS,并消除掉重复着火点,同时满足一系列条件,即得到最终的径迹。
图\ref{fig:ITk_tracking_eff}展示径迹重建效率随$\eta$分布,可达90\%。
值得一提的是,由于径迹重建性能的提高,相比于\RunTwo ,电子电荷误判率下降,见图\ref{fig:ITk_ele_qmisid}。
\begin{figure}[h]
\centering
 \includegraphics[width=0.85\textwidth]{fig/ITk_seeds.png}
 \caption{ITk seeds随\abseta 分布%\footnote{在ITk研究前期,有两种布局,最后选用Inclined作为baseline}
\cite{seeds_ITk}。}
 \label{fig:ITk_seeds}
\end{figure}
\begin{figure}[h]
\centering
\begin{subfigure}[b]{0.45\textwidth}
\centering
 \includegraphics[width=0.9\textwidth]{fig/ITk_tracking_eff_tt.png}
 \caption{}
\end{subfigure}
\begin{subfigure}[b]{0.45\textwidth}
\centering
 \includegraphics[width=0.9\textwidth]{fig/ITk_tracking_eff_ratio.png}
 \caption{}
\end{subfigure}
\caption{(a) 径迹重建效率随$\eta$分布(matching criteria);(b) 重建径迹与真实粒子之比随$\eta$分布(no matching criteria)。\cite{ATL-PHYS-PUB-2016-025}}
\label{fig:ITk_tracking_eff}
\end{figure}
\begin{figure}[h]
\centering
\begin{subfigure}[b]{0.45\textwidth}
\centering
 \includegraphics[width=0.9\textwidth]{fig/ITk_ele_qmisid1.png}
 \caption{}
\end{subfigure}
\begin{subfigure}[b]{0.45\textwidth}
\centering
 \includegraphics[width=0.9\textwidth]{fig/ITk_ele_qmisid2.png}
 \caption{}
\end{subfigure}
\caption{(a)不同工作点的电子鉴别效率随\et 分布;(b)电子电荷误判率随\abseta 分布。\cite{ATL-PHYS-PUB-2019-005}}
\label{fig:ITk_ele_qmisid}
\end{figure}

