\chapter{事例筛选和信号优化}\label{chap:evtsel}

\section{粒子鉴别及筛选}

\subsection{Object definitions in 4W}\label{subsec:4w_obj_def}
粒子鉴别遵循ATLAS一般流程,如章节~\ref{subsec:obj_def}所述。对于4W分析,进一步的粒子筛选条件总结如表~\ref{tab:4w_obj_def}所示。
\begin{table}[h]
\centering
\begin{tabular}{c|c| c c r r}
\hline
\hline
粒子     &\multicolumn{2}{c}{选择条件}    \\
\hline
         &Baseline     &Tight     \\
\cline{2-3}
\multirow{4}{*}{电子}  &\ET$>$ 10 GeV         &\texttt{TightLH} ID      \\
        &\abseta $<$ 2.47,排除1.37$<$\abseta$<$1.52 区间    &\texttt{FixedCutTight} \\
        &\texttt{LooseLH} ID, \texttt{Loose} isolation   &($\et^{\text{cone}20}/\pt<$0.06, $\pt^{\text{varcone}20}/\pt<$0.06)          \\
        &$|z_{0}\sin\theta| <$ 0.5~mm,$d_{0}/\sigma(d_{0}) <$ 5  &   \\
\hline
\multirow{4}{*}{$\mu$}  &\pt > 10 GeV         &\texttt{Tight} ID      \\
        &\abseta $<$ 2.5                      &\texttt{FixedCutTightTrackOnly} \\
        &\texttt{Loose} ID, \texttt{Loose} isolation   &($\pt^{\text{varcone}20}/\pt<$0.06)\\
        &$|z_{0}\sin\theta| <$ 0.5~mm,$d_{0}/\sigma(d_{0}) <$ 3  &   \\
\hline
\multirow{2}{*}{Jet}    &\multicolumn{2}{c}{\pt > 25 GeV, \abseta $<$ 2.5} \\
                        &\multicolumn{2}{c}{ |JVT|<0.59 if \pt < 60 GeV and \abseta $<$ 2.4}  \\
\hline
\met    &\multicolumn{2}{c}{$E_{\text{T}}^{\text{miss,TRK}}$}     \\
\hline
\hline
\end{tabular}
\caption{4W物理分析粒子筛选条件总结}
\label{tab:4w_obj_def}
\end{table}


\subsection{Overlap removal}
经过粒子基准(baseline)筛选之后,为了进一步保证没有误重建的重复粒子,专门的overlap removal需要完成。%~\cite{Adams:1743654}。
该分析中的overlap removal总结如表~\ref{tab:4w_olr}所示。
  \begin{table}
  \centering
  \small
  \begin{tabular}{|c|c|c|}
  \hline
               Keep        &Remove            &Cone size ($\Delta R$) \\
  %\hline
  %             electron    &tau               &0.2  \\
  %\hline
  %             muon        &tau               &0.2  \\
  \hline
               muon    &electron   &0.1 \\
  \hline
               electron     &electron(lower $p_T$)          &0.1 \\
  \hline
               electron    &jet               &0.3      \\
  \hline
               jet        &muon               &min(0.4, 0.04+10[GeV]/$p_T(\mu)$) \\
  \hline
  \end{tabular}
  \caption{Overlap removal in 4W analysis.}
  \label{tab:4w_olr}
  \end{table}


\section{事例筛选}
\subsection{初步筛选}
所选事例应当通过如下初步筛选条件:
\begin{itemize}
  \item \textbf{GRL} \\
	\begin{tabular}{ll}
    2015 data: & {\tt data15\_13TeV.periodAllYear\_DetStatus-v79-repro20-02} \\
                      & {\tt \_DQDefects-00-02-02\_PHYS\_StandardGRL\_All\_Good\_25ns.xml} \\
    %2016 data: & {\tt data16\_13TeV.periodAllYear\_DetStatus-v83-pro20-15} \\
    %                  & {\tt \_DQDefects-00-02-04\_PHYS\_StandardGRL\_All\_Good\_25ns.xml}
   2016 data: & {\tt data16\_13TeV.periodAllYear\_DetStatus-v88-pro20-21} \\
                    & {\tt \_DQDefects-00-02-04\_PHYS\_StandardGRL\_All\_Good\_25ns.xml} \\
  \end{tabular}
  \item \textbf{Event cleaning criteria}: cleaning for Tile corrupted events, LAr noise bursts and corrupted data
  \item \textbf{Vertex criteria}: events are required to contain at least one primary vertex with $\ge2$ associated tracks. The detailed selection on the vertex can be found in~\cite{vertex_ref}
  \item \textbf{Trigger}: \\
  对于2015年数据,满足以下任一trigger:
    \begin{itemize}
    \item Single lepton triggers:
    \begin{itemize}
      \item HLT\_mu20\_iloose\_L1MU15
      \item HLT\_mu50
      \item HLT\_e24\_lhmedium\_L1EM20VH
      \item HLT\_e60\_lhmedium
      \item HLT\_e120\_lhloose
    \end{itemize}
    \item Dilepton triggers:
    \begin{itemize}
      \item HLT\_2e12\_lhloose\_L12EM10VH
      \item HLT\_e17\_lhloose\_mu14
      \item HLT\_mu18\_mu8noL1
    \end{itemize}
  \end{itemize}
  对于2016年数据,满足以下任一trigger:
  \begin{itemize}
    \item Single lepton triggers:
    \begin{itemize}
      \item HLT\_mu24\_ivarmedium
      \item HLT\_mu50
      \item HLT\_e24\_lhtight\_nod0\_ivarloose
      \item HLT\_e60\_lhmedium\_nod0
      \item HLT\_e140\_lhloose\_nod0
    \end{itemize}
    \item Dilepton triggers:
    \begin{itemize}
      \item HLT\_2e17\_lhvloose\_nod0
      \item HLT\_e17\_lhloose\_nod0\_mu14
      \item HLT\_mu22\_mu8noL1
    \end{itemize}
  \end{itemize}
与数据一样,模拟样本也应当满足以上trigger条件,其相应的trigger效率修正已添加到每个样本事例中
  \item 选择通过章节~\ref{subsec:4w_obj_def}所述的粒子。
  \item \textbf{轻子数}:
	\begin{itemize}
	      \item 两个相同电荷的轻子。
	      \item 每个\texttt{tight}电子应当满足\texttt{ChargeIDBDTTight}> 0.067,此变量是用来压低“假”电子,如附录~\ref{app:chargeidbdt}所述。
	      \item 至少有一个轻子应当能匹配以上任一或多个trigger,除此之外,大横动量轻子\pt 应大于30 GeV,小横动量轻子大于20 GeV。
	\end{itemize}
  \item 排除掉任何含有\bjet 的事例。
  \item $E_{\text{T}}^{\text{miss,TRK}}$> 10 GeV。
  \item 因为Drell-Yan过程目前并不能被MC很好模拟,所以为了避免此问题,双轻子不变质量应大于15 GeV。
  \item 为了压低来自于$Z$+jets过程的本底(电荷误判),$|M(\ell\ell)-M(Z)|>$10~GeV条件须通过。
  \item jet数的要求依赖于质量点的选择,低(高)质量点要求至少2(3)个jet。此项会在~\ref{subsec:}深入讨论。
\end{itemize}
以上的事例筛选过程总结在表~\ref{tab:4w_evt_presel}。最后,通过以上筛选条件的事例根据轻子味道分为三个分析道,为$ee$
$\mu\mu$和$e\mu$。表~\ref{tab:presel_cutflow_smhh}展示了标准模型希格斯对信号经过以上一系列条件时的事例数和效率变化,
此处对应亮度为36.1 fb$^{-1}$,截面($gg\rightarrow hh$)为33.4 fb。图~\ref{fig:eff_pre_sel_hh}(图~\ref{fig:eff_pre_sel_SS})展示所有$hh$($SS$)信号样本的经过初步筛选之后的效率,可以看到:一是随着$m_X$或者$m_S$的增加,效率相应增加(对于$SS$,在$m_X=340 GeV, m_S=135 GeV$质量点的效率下降是因为从此点开始要求至少三个jet);二是$e\mu$道具有最高的效率值,$\mu\mu$次之,$ee$最低,这是因为理论上$e\mu$的分支比是其他两个道的两倍,以及$\mu$比$e$具有更好的鉴别效率。
\begin{table}
\centering
\small
\begin{tabular}{c|ccccccccccc}
\hline
\hline
\multirow{12}{2cm}{Pre-selections} &GRL \\
                                  &Event clean criteria\\
                                  &Pass any trigger applied \\
                                  &Select objects following object definitions\\
                                  &Overlap removal \\
                                  &Two tight same-signed leptons, with at least one trigger matched \\
                                  &$p_T(\ell_1) >30$~GeV, $p_T(\ell_2)>20$~GeV \\
                                  &$b$ veto \\
                                  &$E_T^{miss}>$10~GeV \\
                                  &$M(\ell\ell)>$15~GeV \\
                                  &$|M(\ell\ell)-M(Z)|>$10~GeV in $ee$ channel\\
                                  &$N_\text{jet}\geq$2(3) \\
\hline
\hline
\end{tabular}
\caption{4W事例初步筛选条件。}
\label{tab:4w_evt_presel}
\end{table}

\begin{table}
\centering\small
\begin{tabular}{c|ccc|ccc}
\hline
\hline
Cut flow &\multicolumn{3}{c|}{Event yield}&\multicolumn{3}{c}{Efficiency}      \\
\hline
Evgen&\multicolumn{3}{c|}{-}&\multicolumn{3}{c}{100\%}\\
HIGG8D1&\multicolumn{3}{c|}{2.76}&\multicolumn{3}{c}{56.34\%}\\
Event cleaning&\multicolumn{3}{c|}{2.76}&\multicolumn{3}{c}{56.34\%}\\
Trigger&\multicolumn{3}{c|}{2.10    }&\multicolumn{3}{c}{44.84\%}\\
Channel&$ee$&$\mu\mu$&$e\mu$&$ee$&$\mu\mu$&$e\mu$\\
\hline
OB, OLR    &0.29    &0.28    &0.56& 5.86\%    &6.23\%    &11.96\%\\
Tight leptons, trigger match    &0.14    &0.20    &0.33    &2.33\%    &3.46\%    &5.68\%\\
$p_T(\ell)$    &0.11    &0.15    &0.24    &1.93\%    &2.70\%    &4.53\%\\
b veto    &0.10    &0.14    &0.23    &1.79\%    &2.49\%    &4.18\%\\
MET    &0.10    &0.14    &0.22    &1.76\%    &2.45\%    &4.10\%\\
Drell-Yan cut    &0.10    &0.14    &0.22    &1.76\%    &2.44\%    &4.10\%\\
Z veto    &0.08    &0.14    &0.22    &1.58\%    &2.44\%    &4.10\%\\
$N_{\text{jet}}\geq3$    &0.05$\pm$0.002    &0.09$\pm$0.002    &0.14$\pm$0.003    &1.03\%    &1.92\%    &2.99\%\\
\hline
\hline
\end{tabular}
\caption{SM $hh$信号MC的初步筛选效率。结果归一到$\sigma_{\ell^{\pm}\ell^{\pm}}\times \mathcal{L}$,最后一行误差项为统计误差。}
%The cutflow of pre-selection for non-resonant $hh$ signal. The cross-section of $pp \rightarrow hh$ is 33.41 fb. The event yields are normalized to the luminosity of 36.1 fb$^{-1}$, corresponding to the final state of two-signed leptons. The statistical uncertainty is aded in the last row.}
\label{tab:presel_cutflow_smhh}
\end{table}


\begin{figure}
\centering
\includegraphics[width=0.55\textwidth, angle =-90]{fig/SigTopo/eff_presel_hh.pdf}
\caption{The final efficiency of pre-selections for $hh$ signal samples.}
\label{fig:eff_pre_sel_hh}
\end{figure}

\begin{figure}
\centering
\includegraphics[width=0.55\textwidth, angle =-90]{fig/SigTopo/eff_presel_SS.pdf}
\caption{The final efficiency of pre-selections for $SS$ signal samples.}
\label{fig:eff_pre_sel_SS}
\end{figure}

\subsection{信号优化}
\subsubsection{$hh$信号优化}\label{subsubsec:hh_optimization}
本分析中的显著信号是两个相同电荷的轻子和jet数。两个希格斯粒子倾向于出射到两个相反的半球,随后,两个希格斯粒子均衰变
到$W$玻色子,总共四个$W$玻色子中有两个是不在壳的,而不在壳的$W$玻色子会贡献相当部分的低动量的jet(\pt < 25 GeV)。
在图~\ref{fig:SigTopo:pt_jet_mH260}到\ref{fig:SigTopo:pt_jet_nonres}中可以看到(此图未通过初步筛选条件),很大一部分的第四条jet \pt 是低于25 GeV的,甚至在高质量信号点。
那么加上基本的筛选条件之后,大部分信号事例只有三条jet,如图~\ref{fig:SigTopo:numOfjet_sig}所示。同时可以发现,对于低质量点,即$m_X$=260 GeV 和$m_X$=300 GeV,其大部分事例最多只有2条jet。所以,对于不同的质量点,应当应用不同的jet数条件,对于低质量点,要求$N_\text{jet}\ge$2,而高质量点,$N_\text{jet}\ge$3。为了证实该分类能够给出最高的信号显著性,考虑本底后,
详细检查可见附录~\ref{app:}
最后,为了提高信号显著性,一系列动力学被重建,从而用来优化信号,具体优化方法会在章节~\ref{subsec:}具体讨论,
以下列出一些具有区分度的变量:
\begin{itemize}
 \item $M(ll)$, the invariant mass of two same-signed leptons;
 \item $MET$, missing transverse energy;
 \item $M(jj)^W$, the invariant mass of two closest jets among all selected good jets;
 \item $M(l_1jj)$, the invariant mass of leading lepton and two closest jets;
 \item $M(all)$, the invariant mass of all selected objects;
 \item $M_T$, the transverse mass of all selected objects;
 \item $\Delta R_{min}(\ell_1, j)$, $\Delta R$ distance between leading lepton and the closest jet;
 \item $\Delta R_{min}(\ell_2, j)$, $\Delta R$ distance between sub leading lepton and the closest jet;
\end{itemize}

\begin{figure}
\centering
\begin{subfigure}[b]{0.45\textwidth}
 \includegraphics[width=0.75\textwidth, angle =-90]{fig/SigTopo/pt_jet_mH260.pdf}\caption{}
 \label{fig:SigTopo:pt_jet_mH260}
\end{subfigure}
\begin{subfigure}[b]{0.45\textwidth}
 \includegraphics[width=0.75\textwidth, angle =-90]{fig/SigTopo/pt_jet_mH300.pdf}\caption{}
 \label{fig:SigTopo:pt_jet_mH300}
\end{subfigure}
\begin{subfigure}[b]{0.45\textwidth}
 \includegraphics[width=0.75\textwidth, angle =-90]{fig/SigTopo/pt_jet_mH400.pdf}\caption{}
 \label{fig:SigTopo:pt_jet_mH400}
\end{subfigure}
\begin{subfigure}[b]{0.45\textwidth}
 \includegraphics[width=0.75\textwidth, angle =-90]{fig/SigTopo/pt_jet_mH500.pdf}\caption{}
 \label{fig:SigTopo:pt_jet_mH500}
\end{subfigure}
\begin{subfigure}[b]{0.45\textwidth}
 \includegraphics[width=0.75\textwidth, angle =-90]{fig/SigTopo/pt_jet_nonres.pdf}\caption{}
 \label{fig:SigTopo:pt_jet_nonres}
\end{subfigure}
\begin{subfigure}[b]{0.45\textwidth}
 \includegraphics[width=0.75\textwidth, angle =-90]{fig/SigTopo/numOfjet_sig.pdf}\caption{}
 \label{fig:SigTopo:numOfjet_sig}
\end{subfigure}
\caption{Distributions of $p_T$ and number of jet for signal. Figure~\ref{fig:SigTopo:pt_jet_mH260} to Figure~\ref{fig:SigTopo:pt_jet_nonres} are distributions of $p_T$ of jet before 25 GeV cuts, corresponding for mX=260, 300, 400, 500 GeV and non-resonant signal. Two dashed vertical lines are $p_T$=10 GeV and $p_T$=25 GeV, respectively. Figure~\ref{fig:SigTopo:numOfjet_sig} is number of jet distribution after 25 GeV cuts.}
\label{fig:SigTopo:pt_jet}
\end{figure}

\subsubsection{$SS$信号优化}
$S$标量粒子所取质量从135 GeV到165 GeV,$X$粒子从280 GeV到340 GeV。$SS$与$hh$具有类似的动力学性质,为了尽可能增加信号信号显著性,$N_\text{jet}$分类适用于此,具体如下:
\begin{itemize}
 \item 固定$m_S=135$~GeV: $m_X=280$~GeV, $m_X=300$~GeV and $m_X=320$~GeV; $N_{\text{jet}}\geq$2。
 \item 固定$m_X=340$~GeV: $m_S=135$~GeV, $m_S=145$~GeV, $m_S=155$~GeV and $m_S=165$~GeV; $N_{\text{jet}}\geq$3。
\end{itemize}
前述章节~\ref{subsubsec:hh_optimization}的动力学变量也可用来进一步优化信号显著性。
