\section{信号优化}\label{sec:signal_optimization}
在初步筛选之后,为了进一步加强信号显著性,利用运动学性质进行信号优化是有必要的。因为各个信号质量点的运动学性质差别
较大,所以,每个质量点都会进行信号优化过程,而后通过各自运动学优化条件之后的区域才作为每个质量点的最终信号区。

\subsection{优化策略}
MVA方法用于确定不同运动学变量的分离能力,并考虑所有变量之间的相关性。最终,前五个运动学变量用于形成优化选择,分别是$M(\ell\ell)$,$\Delta R_{min}(\ell_{2}, j)$,$\Delta R_{min}(\ell_{1}, j)$ ,$ M_{\ell_{1} jj}$和$M(all)$,它们具有很强的分离能力,而且相互之间的相关性很低(图~\ref{fig:correlation_check})。$M(\ell\ell)$和$ M_{\ell_{1} jj} $对低质量点敏感,而其余对高质量和非共振信号敏感。
基于这些知识,$\Delta R_{min}(\ell_{1}, j)$,$M(\ell\ell)$,$ M_{\ell_{1}jj}$和$M(all)$用于在低质量搜索中,而$\Delta R_{min}(\ell_{2}, j)$,$\Delta R_{min}(\ell_{1}, j)$,$M(\ell\ell)$和$M_{\ell_{1}jj}$用于高质量搜索中。它们的相应分布分别见图~\ref{fig:SigOpt_low_kine}和图~\ref{fig:SigOpt_high_kine}。

TMVA包(CutsSA选项)~\cite{Hocker:2007ht}用于实现最佳筛选。所有本底: promptSS,$V\gamma$,QmisID和fakes都包含在训练中。
为了减少对变量筛选顺序的依赖,每次仅训练2个变量。
对于每个信号效率工作点(WP),在测试样本中应用对应的选择条件,
并计算显著性($S/\sqrt{B}$)。随后,选择具有最高信号显著性的WP,对应该WP的2个变量筛选值即为最佳选择,
最后再对剩下两个变量重复以上步骤。图~\ref{fig:nonres:SigOpt_mumu}展示SM希格斯粒子对搜寻中$\mu\mu$分析道效率变化,各个运动学
变量的最佳选择上下限以及显著性随信号效率WP的分布。
对剩余的分析道或者其他质量点重复此操作,即可得到所有质量点的不同味道子类的最佳优化选择条件。值得指出的是,对于$SS$信号优化,因为各个质量点之间的运动学性质比较接近,所以只针对$m_S=135$~GeV, $m_X=300$~GeV ($m_X=340$~GeV, $m_S=145$~GeV)进行优化,而后应用在所有的低(高)质量点。 

最终考虑从低到高(和非共振)质量点筛选值的单调性,一定的选择调整被执行。最终的选择总结在表~\ref{optimization_cuts_lowmass},表~\ref {optimization_cuts_highmass}和表~\ref{optimization_cuts_HSS}中,分别对应于$hh$低质量,$hh$高质量和$SS$信号寻找。

\begin{figure}[h]
\centering
\includegraphics[width = 0.4\textwidth,angle=-90]{fig/SigOpt/correlation_signal.pdf}
\includegraphics[width = 0.4\textwidth,angle=-90]{fig/SigOpt/correlation_bkg.pdf}
\caption{训练变量之间的相关性。}
%\caption{Correlation check of input training variables.} \label{fig:correlation_check}
\label{fig:correlation_check}
\end{figure}

\begin{figure}[h]
\begin{minipage}[t]{0.33\linewidth}
 \centering
 \includegraphics[width=0.9\textwidth,angle=-90]{fig/dataMC_low_Njet_CR/mindR_l1j_ee.pdf}\label{fig:dataMC_low_Njet_CR:mindRl1j_ee.pdf}
 \end{minipage}
 \begin{minipage}[t]{0.33\linewidth}
 \centering
 \includegraphics[width=0.9\textwidth,angle=-90]{fig/dataMC_low_Njet_CR/mindR_l1j_mumu.pdf}\label{fig:dataMC_low_Njet_CR:mindRl1j_mumu.pdf}
 \end{minipage}
 \begin{minipage}[t]{0.33\linewidth}
 \centering
 \includegraphics[width=0.9\textwidth,angle=-90]{fig/dataMC_low_Njet_CR/mindR_l1j_emu.pdf}\label{fig:dataMC_low_Njet_CR:mindRl1j_emu.pdf}
 \end{minipage}
 \begin{minipage}[t]{0.33\linewidth}
 \centering
 \includegraphics[width=0.9\textwidth,angle=-90]{fig/dataMC_low_Njet_CR/m_ll_ee.pdf}
 \label{fig:dataMC_low_Njet_CR:m_ll_ee.pdf}
 \end{minipage}
 \begin{minipage}[t]{0.33\linewidth}
 \centering
 \includegraphics[width=0.9\textwidth,angle=-90]{fig/dataMC_low_Njet_CR/m_ll_mumu.pdf}
 \label{fig:dataMC_low_Njet_CR:m_ll_mumu.pdf}
 \end{minipage}
  \begin{minipage}[t]{0.33\linewidth}
 \centering
 \includegraphics[width=0.9\textwidth,angle=-90]{fig/dataMC_low_Njet_CR/m_ll_emu.pdf}
 \label{fig:dataMC_low_Njet_CR:m_ll_emu.pdf}
 \end{minipage}
\begin{minipage}[t]{0.33\linewidth}
 \centering
 \includegraphics[width=0.9\textwidth,angle=-90]{fig/dataMC_low_Njet_CR/m_l1jj_ee.pdf}\label{fig:dataMC_low_Njet_CR:m_l1jj_ee.pdf}
 \end{minipage}
 \begin{minipage}[t]{0.33\linewidth}
 \centering
 \includegraphics[width=0.9\textwidth,angle=-90]{fig/dataMC_low_Njet_CR/m_l1jj_mumu.pdf}\label{fig:dataMC_low_Njet_CR:m_l1jj_mumu.pdf}
 \end{minipage}
 \begin{minipage}[t]{0.33\linewidth}
 \centering
 \includegraphics[width=0.9\textwidth,angle=-90]{fig/dataMC_low_Njet_CR/m_l1jj_emu.pdf}\label{fig:dataMC_low_Njet_CR:m_l1jj_emu.pdf}
 \end{minipage}
 \begin{minipage}[t]{0.33\linewidth}
 \centering
 \includegraphics[width=0.9\textwidth,angle=-90]{fig/dataMC_low_Njet_CR/m_all_ee.pdf}\label{fig:dataMC_low_Njet_CR:m_all_ee.pdf}
 \end{minipage}
  \begin{minipage}[t]{0.33\linewidth}
 \centering
 \includegraphics[width=0.9\textwidth,angle=-90]{fig/dataMC_low_Njet_CR/m_all_mumu.pdf}\label{fig:dataMC_low_Njet_CR:m_all_mumu.pdf}
 \end{minipage}
 \begin{minipage}[t]{0.33\linewidth}
 \centering
 \includegraphics[width=0.9\textwidth,angle=-90]{fig/dataMC_low_Njet_CR/m_all_emu.pdf}\label{fig:dataMC_low_Njet_CR:m_all_emu.pdf}
 \end{minipage}
 \caption{用于低质量信号优化的运动学变量分布,对应$N_{\text{jet}}\geq2$。左: $ee$,中: $\mu\mu$,右: $e\mu$。}
 %\caption{The distributions of kinematic variables that are used to form optimization selections at pre-selection level, corresponding to $N_{\text{jet}}\geq2$. Left: $ee$, middle: $\mu\mu$, right: $e\mu$. PromptSS and $V+\gamma$ are normalized to the luminosity of 36.1 fb$^{-1}$.}
\label{fig:SigOpt_low_kine}
\end{figure}

\begin{figure}[h]
 \begin{minipage}[t]{0.33\linewidth}
 \centering
 \includegraphics[width=0.9\textwidth,angle=-90]{fig/dataMC_high_Njet_CR/mindR_l2j_ee.pdf}\label{fig:dataMC_high_Njet_CR:mindRl2j_ee.pdf}
 \end{minipage}
 \begin{minipage}[t]{0.33\linewidth}
 \centering
 \includegraphics[width=0.9\textwidth,angle=-90]{fig/dataMC_high_Njet_CR/mindR_l2j_mumu.pdf}\label{fig:dataMC_high_Njet_CR:mindRl2j_mumu.pdf}
 \end{minipage}
 \begin{minipage}[t]{0.33\linewidth}
 \centering
 \includegraphics[width=0.9\textwidth,angle=-90]{fig/dataMC_high_Njet_CR/mindR_l2j_emu.pdf}\label{fig:dataMC_high_Njet_CR:mindRl2j_emu.pdf}
 \end{minipage}
 \begin{minipage}[t]{0.33\linewidth}
 \centering
 \includegraphics[width=0.9\textwidth,angle=-90]{fig/dataMC_high_Njet_CR/mindR_l1j_ee.pdf}\label{fig:dataMC_high_Njet_CR:mindRl1j_ee.pdf}
 \end{minipage}
 \begin{minipage}[t]{0.33\linewidth}
 \centering
 \includegraphics[width=0.9\textwidth,angle=-90]{fig/dataMC_high_Njet_CR/mindR_l1j_mumu.pdf}\label{fig:dataMC_high_Njet_CR:mindRl1j_mumu.pdf}
 \end{minipage}
  \begin{minipage}[t]{0.33\linewidth}
 \centering
 \includegraphics[width=0.9\textwidth,angle=-90]{fig/dataMC_high_Njet_CR/mindR_l1j_emu.pdf}\label{fig:dataMC_high_Njet_CR:mindRl1j_emu.pdf}
 \end{minipage}
\begin{minipage}[t]{0.33\linewidth}
 \centering
 \includegraphics[width=0.9\textwidth,angle=-90]{fig/dataMC_high_Njet_CR/m_ll_ee.pdf}
 \label{fig:dataMC_high_Njet_CR:m_ll_ee.pdf}
 \end{minipage}
 \begin{minipage}[t]{0.33\linewidth}
 \centering
 \includegraphics[width=0.9\textwidth,angle=-90]{fig/dataMC_high_Njet_CR/m_ll_mumu.pdf}
 \label{fig:dataMC_high_Njet_CR:m_ll_mumu.pdf}
 \end{minipage}
 \begin{minipage}[t]{0.33\linewidth}
 \centering
 \includegraphics[width=0.9\textwidth,angle=-90]{fig/dataMC_high_Njet_CR/m_ll_emu.pdf}
 \label{fig:dataMC_high_Njet_CR:m_ll_emu.pdf}
 \end{minipage}
\begin{minipage}[t]{0.33\linewidth}
 \centering
 \includegraphics[width=0.9\textwidth,angle=-90]{fig/dataMC_high_Njet_CR/m_l1jj_ee.pdf}\label{fig:dataMC_high_Njet_CR:m_l1jj_ee.pdf}
 \end{minipage}
  \begin{minipage}[t]{0.33\linewidth}
 \centering
 \includegraphics[width=0.9\textwidth,angle=-90]{fig/dataMC_high_Njet_CR/m_l1jj_mumu.pdf}\label{fig:dataMC_high_Njet_CR:m_l1jj_mumu.pdf}
 \end{minipage}
 \begin{minipage}[t]{0.33\linewidth}
 \centering
 \includegraphics[width=0.9\textwidth,angle=-90]{fig/dataMC_high_Njet_CR/m_l1jj_emu.pdf}\label{fig:dataMC_high_Njet_CR:m_l1jj_emu.pdf}
 \end{minipage}
 \caption{用于低质量信号优化的运动学变量分布,对应$N_{\text{jet}}\geq3$。左: $ee$,中: $\mu\mu$,右: $e\mu$。}
 %\caption{The distributions of kinematic variables that are used to form optimization selections at pre-selection level, corresponding to $N_{\text{jet}}\geq3$. Left: $ee$, middle: $\mu\mu$, right: $e\mu$. PromptSS and $V+\gamma$ are normalized to the luminosity of 36.1 fb$^{-1}$.}
\label{fig:SigOpt_high_kine}
\end{figure}
\begin{figure}[h]
\begin{center}
\includegraphics[width = 0.3\textwidth,angle=-90]{fig/SigOpt/nonres/Efficiency_mumu.pdf}
\includegraphics[width = 0.3\textwidth,angle=-90]{fig/SigOpt/nonres/mindR_l2j_mumu.pdf}
\includegraphics[width = 0.3\textwidth,angle=-90]{fig/SigOpt/nonres/mindR_l1j_mumu.pdf}
\includegraphics[width = 0.3\textwidth,angle=-90]{fig/SigOpt/nonres/m_ll_mumu.pdf}
\includegraphics[width = 0.3\textwidth,angle=-90]{fig/SigOpt/nonres/m_l1jj_mumu.pdf}
\includegraphics[width = 0.3\textwidth,angle=-90]{fig/SigOpt/nonres/SOverRootB_mumu.pdf}
\caption{左上:信号效率和本底效率变化,右下: SM $hh$~$\mu\mu$信号显著性随给定效率分布情况,其他:运动学变化最佳筛选上下限随信号效率变化情况。显著性计算的信号与本底均考虑了统计误差,0.72选作SM $hh$~$\mu\mu$的最佳工作点,其相应的运动学变量筛选值则用作最终信号优化条件(会进行一定的平滑性选择)。}
%\caption{The significance scan as a function of efficiency for $\mu\mu$ in non-resonant signal search. Statistical uncertainties on the background and the signal are considered. The 0.72 working point is chosen for $\mu\mu$ channel in the non-resonant signal optimizations.}
\label{fig:nonres:SigOpt_mumu}
\end{center}
\end{figure}

\begin{table}[h]
\begin{center}
\begin{tabular}{c|c|c|c|c|c}
\hline
\hline
  &Channel &$\Delta R_{min}(\ell_{1}, j)$ &$M(ll)$  &$M_{\ell_{1}jj}$ &$M(all)$\\
\hline
\multirow{3}{2.0cm}{$m_X$=260 GeV} &$ee$  &0.35, 1.85
&$<$100
&$<$145
&$<$1100  \\
&$\mu\mu$
&0.25, 2.10
&$<$80
&$<$115
&$<$700 \\
&$e\mu$
&0.25, 1.80
&$<$85
&$<$135
&$<$650\\
\hline
\multirow{3}{2.0cm}{$m_X$=300 GeV} &$ee$
&0.35, 1.75
&$<$120
&$<$160
&$<$1400 \\
&$\mu\mu$
&0.20, 1.75
&$<$115
&$<$185
&$<$1000 \\
&$e\mu$
&0.20, 1.80
&$<$135
&$<$160
&$<$800 \\
\hline
\hline
\end{tabular}
\end{center}
\caption{基于运动学变量$X \rightarrow hh$信号优化总结,对应$m_{X}$=260, 300 GeV,所有不变质量单位为GeV。}
%\caption{Summary of optimization selections for the search of $X \rightarrow hh$ ($m_{X}$=260, 300 GeV). All mass cuts are in GeV.}
\label{optimization_cuts_lowmass}
\end{table}

\begin{table}[h]
\begin{center}
\begin{tabular}{c|c|c|c|c|c}
\hline
\hline
  &Channel &$\Delta R_{min}(\ell_{2}, j)$ &$\Delta R_{min}(\ell_{1}, j)$ &$M(ll)$  &$M_{\ell_{1}jj}$ \\
\hline
\multirow{3}{2.0cm}{$m_X$=400 GeV} &$ee$
&0.35, 1.50
&0.30, 1.25
&45, 235
&40, 285 \\
&$\mu\mu$
&0.20, 1.20
&0.20, 1.20
&40, 215
&30, 260 \\
&$e\mu$
&0.20, 1.50
&0.20, 1.05
&35, 195
&30, 235 \\
\hline
\multirow{3}{2.0cm}{$m_X$=500 GeV} &$ee$
&0.20, 1.15
&0.20, 1.15
&100, 270
&40, 285 \\
&$\mu\mu$
&0.20, 1.05
&0.20, 0.75
&60, 250
&30, 310 \\
&$e\mu$
&0.20, 1.00
&0.20, 0.80
&75, 250
&35, 350 \\
\hline
\multirow{3}{2.0cm}{Non-resonant} & $ee$
&0.20, 1.40
&0.20, 1.15
&55, 270
&40, 285 \\
&$\mu\mu$
&0.20, 1.05
&0.20, 0.75
&60, 250
&30, 310 \\
&$e\mu$
&0.20, 1.15
&0.20, 0.80
&75, 250
&35, 350 \\
\hline
\hline
\end{tabular}
\end{center}
\caption{基于运动学变量$X \rightarrow hh$信号优化总结,对应$m_{X}$=400, 500 GeV以及SM $hh$,所有不变质量单位为GeV。}
%\caption{Summary of optimization selections for the search of $X \rightarrow hh$($m_{X}$=400, 500 GeV and non-resonant). All mass cuts are in GeV.}
\label{optimization_cuts_highmass}
\end{table}
 
\begin{table}[h]
\begin{center}
\begin{tabular}{c|c|c|c|c|c}
\hline
\hline
  &Channel &$\Delta R_{min}(\ell_{2}, j)$ &$\Delta R_{min}(\ell_{1}, j)$ &$M(ll)$  &$M_{\ell_{1}jj}$ \\
%\hline
%\multirow{3}{1.5cm}{mX=280 GeV, mS=135 GeV} &$ee$
%0.2 2.5
%0.2 1.7
%15 70
%40 125 \\
%&$\mu\mu$
%0.2 2
%0.25 1.65
%15 90
%0 130 \\
%&$e\mu$
%0.25 2.15
%0.2 1.7
%15 95
%0 120 \\
\hline
\multirow{3}{4.5cm}{$m_X$=300 GeV, $m_S$=135 GeV} &$ee$
&0.35, 2.5
&0.4, 1.65
& $<$80
&50, 150 \\
&$\mu\mu$
&0.25, 2.05
&0.2, 1.85
&$<$ 95
&50, 150 \\
&$e\mu$
&0.25, 1.7
&0.25, 1.65
&$<$ 95
&50, 150 \\
\hline
\multirow{3}{4.5cm}{$m_X$=340 GeV, $m_S$=145 GeV} &$ee$
&0.35, 1.85
&0.2, 1.65
&$<$ 130
&50, 190 \\
&$\mu\mu$
&0.2, 2.0
&0.2, 1.65
&$<$ 115
&50, 185 \\
&$e\mu$
&0.25, 1.6
&0.25, 1.6
&$<$ 150
&50, 150 \\
\hline
\hline
\end{tabular}
\end{center}
\caption{Summary of optimization selections for the search of $X\rightarrow SS$. All mass cuts are in GeV.}
\label{optimization_cuts_HSS}
\end{table}

 
\clearpage
\subsection{优化效率检查}
为了防止过度优化或者欠优化的情况,可以检查每个信号MC经过以上选择条件之后的效率。
图~\ref {fig:eff_sigopt_hh}和图~\ref {fig:eff_sigopt_SS}分别表示$hh$和$SS$信号相对于经过初步筛选条件之后的选择效率。 总体上大多数质量点的选择效率相当接近,
 但是由于优化时每两个变量一组,它们之间的相关性在三个分析道中略有不同,并且各个分析道具有不同的背景组成,导致不同质量点不同分析道具有不同效率的趋势。
 \begin{figure}[h]
\begin{center}
\includegraphics[width = 0.45\textwidth,angle=-90]{fig/SigOpt/eff_sigopt_hh.pdf}
\caption{$hh$信号优化效率(相比于初步筛选)。}
%\caption{Signal efficiency with respect to pre-selections in $ee$, $\mu\mu$ and $e\mu$ channel after applying all of the optimisation selections
%for the non-resonant and resonant $hh$ signal.}
\label{fig:eff_sigopt_hh}
\end{center}
\end{figure}

\begin{figure}[h]
\begin{center}
\includegraphics[width = 0.50\textwidth,angle=-90]{fig/SigOpt/eff_sigopt_SS.pdf}
\caption{$SS$信号优化效率(相比于初步筛选)。}
%\caption{Signal efficiency with respect to pre-selections in $ee$, $\mu\mu$ and $e\mu$ channel after applying all of the optimisation selections
%for the resonant $SS$ signal.}
\label{fig:eff_sigopt_SS}
\end{center}
\end{figure}

\clearpage
\subsection{优化结果}
\label{sec:unblind_results}
经过所有筛选条件之后,表~\ref{cutflow_ee_nonres}到表~\ref{cutflow_emu_HSS_highmass}展示各个质量点不同分析道中的各种背景,预期信号以及观测事例数的结果。总预期本底的误差包括了所有
的系统误差,并考虑了非fakes本底的系统误差(syst1)与fakes本底的系统误差(syst2)的反相关性质。
图~\ref{fig:SigOpt:nonres_m_l1jj.pdf}到图~\ref{fig:SigOpt:H340_S165_m_l1jj.pdf}展示对应的子类的运动学分布。
总体而言,所有子类中没有发现明显超出。
\subsubsection{$hh$搜寻筛选结果}
\begin{table}[h]
\begin{center}
\tiny\scalebox{0.65}{
\begin{tabular}{c|cccc|cc|c}
\hline
\hline
                           &promptSS  &$V+\gamma$  &QmisID  &Fakes  &Total bkg  &Observed  &signal  \\  
\hline
$\Delta R_{min}(l_{2},j)$ &46.87$\pm$2.91(stat.)$\pm$14.06(syst.)  &16.25$\pm$3.99(stat.)$\pm$8.12(syst.)  &15.60$\pm$0.24(stat.)$\pm$5.15(syst.)  &64.87$\pm$5.96(stat.)$\pm$44.88(syst.)  &143.59$\pm$7.74(stat.)$\mp$17.04(syst1.)$\pm$44.88(syst2.)  &158  &0.04$\pm$0.00\\
\hline
$\Delta R_{min}(l_{1},j)$ &16.38$\pm$1.80(stat.)$\pm$4.91(syst.)  &2.89$\pm$1.04(stat.)$\pm$1.45(syst.)  &5.23$\pm$0.13(stat.)$\pm$1.72(syst.)  &26.32$\pm$3.79(stat.)$\pm$18.21(syst.)  &50.81$\pm$4.33(stat.)$\mp$5.40(syst1.)$\pm$18.21(syst2.)  &62  &0.03$\pm$0.00\\
\hline
$M(\ell\ell)$ &11.70$\pm$1.65(stat.)$\pm$3.51(syst.)  &0.95$\pm$0.34(stat.)$\pm$0.48(syst.)  &3.38$\pm$0.10(stat.)$\pm$1.12(syst.)  &21.24$\pm$3.41(stat.)$\pm$14.70(syst.)  &37.28$\pm$3.80(stat.)$\mp$3.72(syst1.)$\pm$14.70(syst2.)  &46  &0.03$\pm$0.00\\
\hline
$M(l_{1}jj)$ &8.37$\pm$1.04(stat.)$\pm$2.51(syst.)  &0.54$\pm$0.24(stat.)$\pm$0.27(syst.)  &2.61$\pm$0.09(stat.)$\pm$0.86(syst.)  &17.46$\pm$3.09(stat.)$\pm$12.08(syst.)  &28.98$\pm$3.27(stat.)$\mp$2.67(syst1.)$\pm$12.08(syst2.)  &35  &0.03$\pm$0.00\\
\hline
\hline
\end{tabular}}
\end{center}

\caption{SM $hh$ $ee$类别的优化结果。 }
\label{cutflow_ee_nonres}
\end{table}

\begin{table}[h]
\begin{center}
\tiny\scalebox{0.8}{
\begin{tabular}{c|cccc|cc|c}
\hline
\hline
                           &promptSS  &$V+\gamma$  &QmisID  &Fakes  &Total bkg  &Observed  &signal  \\  
\hline
$\Delta R_{min}(l_{2},j)$ &47.41$\pm$2.70(stat.)$\pm$14.22(syst.)  &0.01$\pm$0.01(stat.)$\pm$0.00(syst.)  &0.00$\pm$0.00(stat.)$\pm$0.00(syst.)  &37.76$\pm$4.14(stat.)$\pm$27.11(syst.)  &85.17$\pm$4.94(stat.)$\mp$14.22(syst1.)$\pm$27.11(syst2.)  &72  &0.07$\pm$0.00\\
\hline
$\Delta R_{min}(l_{1},j)$ &9.52$\pm$1.17(stat.)$\pm$2.86(syst.)  &0.00$\pm$0.00(stat.)$\pm$0.00(syst.)  &0.00$\pm$0.00(stat.)$\pm$0.00(syst.)  &5.59$\pm$1.59(stat.)$\pm$4.01(syst.)  &15.11$\pm$1.98(stat.)$\mp$2.86(syst1.)$\pm$4.01(syst2.)  &10  &0.04$\pm$0.00\\
\hline
$M(\ell\ell)$ &6.21$\pm$0.97(stat.)$\pm$1.86(syst.)  &0.00$\pm$0.00(stat.)$\pm$0.00(syst.)  &0.00$\pm$0.00(stat.)$\pm$0.00(syst.)  &4.01$\pm$1.35(stat.)$\pm$2.88(syst.)  &10.22$\pm$1.66(stat.)$\mp$1.86(syst1.)$\pm$2.88(syst2.)  &4  &0.04$\pm$0.00\\
\hline
$M(l_{1}jj)$ &4.50$\pm$0.74(stat.)$\pm$1.35(syst.)  &0.00$\pm$0.00(stat.)$\pm$0.00(syst.)  &0.00$\pm$0.00(stat.)$\pm$0.00(syst.)  &3.56$\pm$1.27(stat.)$\pm$2.55(syst.)  &8.05$\pm$1.47(stat.)$\mp$1.35(syst1.)$\pm$2.55(syst2.)  &4  &0.03$\pm$0.00\\
\hline
\hline
\end{tabular}}
\end{center}

\caption{SM $hh$ $\mu\mu$类别的优化结果。}
%\caption{The unblinded results of non-resonant search in $\mu\mu$ channel. }
\label{cutflow_mumu_nonres}
\end{table}

\begin{table}[h]
\begin{center}
\tiny\scalebox{0.65}{
\begin{tabular}{c|cccc|cc|c}
\hline
\hline
                           &promptSS  &$V+\gamma$  &QmisID  &Fakes  &Total bkg  &Observed  &signal  \\  
\hline
$\Delta R_{min}(l_{2},j)$ &94.91$\pm$3.97(stat.)$\pm$28.47(syst.)  &15.89$\pm$4.14(stat.)$\pm$7.95(syst.)  &3.46$\pm$0.11(stat.)$\pm$1.14(syst.)  &48.27$\pm$4.96(stat.)$\pm$24.35(syst.)  &162.53$\pm$7.59(stat.)$\mp$29.58(syst1.)$\pm$24.35(syst2.)  &194  &0.11$\pm$0.00\\
\hline
$\Delta R_{min}(l_{1},j)$ &19.61$\pm$1.80(stat.)$\pm$5.88(syst.)  &1.88$\pm$0.94(stat.)$\pm$0.94(syst.)  &0.68$\pm$0.05(stat.)$\pm$0.22(syst.)  &9.16$\pm$2.20(stat.)$\pm$5.30(syst.)  &31.33$\pm$2.99(stat.)$\mp$5.96(syst1.)$\pm$5.30(syst2.)  &44  &0.06$\pm$0.00\\
\hline
$M(\ell\ell)$ &11.34$\pm$1.29(stat.)$\pm$3.40(syst.)  &0.24$\pm$0.21(stat.)$\pm$0.12(syst.)  &0.34$\pm$0.03(stat.)$\pm$0.11(syst.)  &1.73$\pm$0.94(stat.)$\pm$0.89(syst.)  &13.65$\pm$1.61(stat.)$\mp$3.41(syst1.)$\pm$0.89(syst2.)  &21  &0.05$\pm$0.00\\
\hline
$M(l_{1}jj)$ &9.28$\pm$1.15(stat.)$\pm$2.79(syst.)  &0.24$\pm$0.21(stat.)$\pm$0.12(syst.)  &0.27$\pm$0.03(stat.)$\pm$0.09(syst.)  &1.33$\pm$0.82(stat.)$\pm$0.66(syst.)  &11.13$\pm$1.43(stat.)$\mp$2.79(syst1.)$\pm$0.66(syst2.)  &18  &0.05$\pm$0.00\\
\hline
\hline
\end{tabular}}
\end{center}

\caption{SM $hh$ $e\mu$类别的优化结果。}
%\caption{The unblinded results of non-resonant search in $e\mu$ channel. }
\label{cutflow_emu_nonres}
\end{table}

\begin{figure}[h]
\begin{minipage}[t]{0.33\linewidth}
 \centering
 \includegraphics[width=0.9\textwidth,angle=-90]{fig/SigOpt/nonres_m_l1jj_ee.pdf}
 \end{minipage}
 \begin{minipage}[t]{0.33\linewidth}
 \centering
 \includegraphics[width=0.9\textwidth,angle=-90]{fig/SigOpt/nonres_m_l1jj_mumu.pdf}
 \end{minipage}
 \begin{minipage}[t]{0.33\linewidth}
 \centering
 \includegraphics[width=0.9\textwidth,angle=-90]{fig/SigOpt/nonres_m_l1jj_emu.pdf}
 \end{minipage}
 \caption{SM hh质量点经过优化筛选之后$M(\ell_{1}jj)$分布。}
%\caption{The unblinded $M(\ell_{1}jj)$ distribution after all optimization selections, corresponding to non-resonance search.}
\label{fig:SigOpt:nonres_m_l1jj.pdf}
\end{figure}

\begin{table}[h]
\begin{center}
\tiny\scalebox{0.8}{
\begin{tabular}{c|cccc|cc|c}
\hline
\hline
                          &promptSS  &$V+\gamma$  &QmisID  &Fakes  &Total bkg  &Observed  &signal  \\
\hline
$\Delta R_{min}(l_{2},j)$ &110.96$\pm$4.65(stat.)$\pm$33.29(syst.)  &37.23$\pm$6.02(stat.)$\pm$18.61(syst.)  &39.43$\pm$0.35(stat.)$\pm$13.01(syst.)  &145.41$\pm$8.86(stat.)$\pm$91.74(syst.)  &333.03$\pm$11.69(stat.)$\mp$40.30(syst1.)$\pm$91.74(syst2.)  &371  &0.59$\pm$0.03\\
\hline
$\Delta R_{min}(l_{1},j)$ &39.91$\pm$2.89(stat.)$\pm$11.97(syst.)  &21.34$\pm$4.74(stat.)$\pm$10.67(syst.)  &17.43$\pm$0.18(stat.)$\pm$5.75(syst.)  &86.67$\pm$6.84(stat.)$\pm$54.68(syst.)  &165.35$\pm$8.81(stat.)$\mp$17.04(syst1.)$\pm$54.68(syst2.)  &173  &0.58$\pm$0.03\\
\hline
$M(\ell\ell)$ &12.41$\pm$1.74(stat.)$\pm$3.72(syst.)  &3.34$\pm$1.26(stat.)$\pm$1.67(syst.)  &4.78$\pm$0.06(stat.)$\pm$1.58(syst.)  &28.20$\pm$3.90(stat.)$\pm$17.79(syst.)  &48.72$\pm$4.45(stat.)$\mp$4.37(syst1.)$\pm$17.79(syst2.)  &63  &0.44$\pm$0.03\\
\hline
$M(l_{1}jj)$ &11.71$\pm$1.71(stat.)$\pm$3.51(syst.)  &3.34$\pm$1.26(stat.)$\pm$1.67(syst.)  &4.68$\pm$0.06(stat.)$\pm$1.55(syst.)  &27.96$\pm$3.89(stat.)$\pm$17.64(syst.)  &47.70$\pm$4.43(stat.)$\mp$4.19(syst1.)$\pm$17.64(syst2.)  &62  &0.44$\pm$0.02\\
\hline
\hline
\end{tabular}}
\end{center}

\caption{$m_X$=260~GeV $ee$类别的优化结果。}
%\caption{The unblinded results of $m_X$=260~GeV search in $ee$ channel. }
\label{cutflow_ee_mX260}
\end{table}

\begin{table}[h]
\begin{center}
\tiny\scalebox{0.65}{
\begin{tabular}{c|cccc|cc|c}
\hline
\hline
                          &promptSS  &$V+\gamma$  &QmisID  &Fakes  &Total bkg  &Observed  &signal  \\
\hline
$\Delta R_{min}(l_{2},j)$ &207.37$\pm$6.74(stat.)$\pm$62.21(syst.)  &0.01$\pm$0.01(stat.)$\pm$0.00(syst.)  &0.00$\pm$0.00(stat.)$\pm$0.00(syst.)  &181.81$\pm$9.57(stat.)$\pm$117.36(syst.)  &389.18$\pm$11.70(stat.)$\mp$62.21(syst1.)$\pm$117.36(syst2.)  &309  &1.21$\pm$0.04\\
\hline
$\Delta R_{min}(l_{1},j)$ &73.92$\pm$4.34(stat.)$\pm$22.18(syst.)  &0.00$\pm$0.00(stat.)$\pm$0.00(syst.)  &0.00$\pm$0.00(stat.)$\pm$0.00(syst.)  &91.31$\pm$6.78(stat.)$\pm$58.94(syst.)  &165.23$\pm$8.05(stat.)$\mp$22.18(syst1.)$\pm$58.94(syst2.)  &102  &1.07$\pm$0.04\\
\hline
$M(\ell\ell)$ &10.34$\pm$1.52(stat.)$\pm$3.10(syst.)  &0.00$\pm$0.00(stat.)$\pm$0.00(syst.)  &0.00$\pm$0.00(stat.)$\pm$0.00(syst.)  &17.80$\pm$2.99(stat.)$\pm$11.49(syst.)  &28.13$\pm$3.36(stat.)$\mp$3.10(syst1.)$\pm$11.49(syst2.)  &20  &0.56$\pm$0.03\\
\hline
$M(l_{1}jj)$ &8.79$\pm$1.47(stat.)$\pm$2.64(syst.)  &0.00$\pm$0.00(stat.)$\pm$0.00(syst.)  &0.00$\pm$0.00(stat.)$\pm$0.00(syst.)  &14.91$\pm$2.74(stat.)$\pm$9.63(syst.)  &23.70$\pm$3.11(stat.)$\mp$2.64(syst1.)$\pm$9.63(syst2.)  &17  &0.54$\pm$0.03\\
\hline
\hline
\end{tabular}}
\end{center}

\caption{$m_X$=260~GeV $\mu\mu$类别的优化结果。}
%\caption{The unblinded results of $m_X$=260~GeV search in $\mu\mu$ channel. }
\label{cutflow_mumu_mX260}
\end{table}

\begin{table}[h]
\begin{center}
\tiny\scalebox{0.65}{
\begin{tabular}{c|cccc|cc|c}
\hline
\hline
                          &promptSS  &$V+\gamma$  &QmisID  &Fakes  &Total bkg  &Observed  &signal  \\
\hline
$\Delta R_{min}(l_{2},j)$ &282.43$\pm$7.10(stat.)$\pm$84.73(syst.)  &36.01$\pm$5.38(stat.)$\pm$18.00(syst.)  &9.42$\pm$0.16(stat.)$\pm$3.11(syst.)  &169.37$\pm$9.45(stat.)$\pm$78.88(syst.)  &497.22$\pm$12.99(stat.)$\mp$86.67(syst1.)$\pm$78.88(syst2.)  &589  &1.69$\pm$0.05\\
\hline
$\Delta R_{min}(l_{1},j)$ &105.47$\pm$4.54(stat.)$\pm$31.64(syst.)  &18.73$\pm$4.47(stat.)$\pm$9.37(syst.)  &2.02$\pm$0.04(stat.)$\pm$0.67(syst.)  &103.33$\pm$7.37(stat.)$\pm$47.60(syst.)  &229.55$\pm$9.74(stat.)$\mp$33.00(syst1.)$\pm$47.60(syst2.)  &244  &1.53$\pm$0.05\\
\hline
$M(\ell\ell)$ &29.17$\pm$2.42(stat.)$\pm$8.75(syst.)  &5.89$\pm$1.87(stat.)$\pm$2.95(syst.)  &0.54$\pm$0.02(stat.)$\pm$0.18(syst.)  &45.62$\pm$4.90(stat.)$\pm$20.94(syst.)  &81.23$\pm$5.77(stat.)$\mp$9.23(syst1.)$\pm$20.94(syst2.)  &80  &1.04$\pm$0.04\\
\hline
$M(l_{1}jj)$ &23.61$\pm$2.14(stat.)$\pm$7.08(syst.)  &4.87$\pm$1.67(stat.)$\pm$2.44(syst.)  &0.46$\pm$0.01(stat.)$\pm$0.15(syst.)  &41.89$\pm$4.69(stat.)$\pm$19.28(syst.)  &70.84$\pm$5.42(stat.)$\mp$7.49(syst1.)$\pm$19.28(syst2.)  &70  &0.99$\pm$0.04\\
\hline
\hline
\end{tabular}}
\end{center}

\caption{$m_X$=260~GeV $e\mu$类别的优化结果。}
%\caption{The unblinded results of $m_X$=260~GeV search in $e\mu$ channel. }
\label{cutflow_emu_mX260}
\end{table}

\begin{figure}[h]
\begin{minipage}[t]{0.33\linewidth}
 \centering
 \includegraphics[width=0.9\textwidth,angle=-90]{fig/SigOpt/mH260_m_all_ee.pdf}
 \end{minipage}
 \begin{minipage}[t]{0.33\linewidth}
 \centering
 \includegraphics[width=0.9\textwidth,angle=-90]{fig/SigOpt/mH260_m_all_mumu.pdf}
 \end{minipage}
 \begin{minipage}[t]{0.33\linewidth}
 \centering
 \includegraphics[width=0.9\textwidth,angle=-90]{fig/SigOpt/mH260_m_all_emu.pdf}
 \end{minipage}
 \caption{$m_X$=260 GeV质量点经过优化筛选之后$M(all)$分布。}
%\caption{The unblinded $M(all)$ distribution after all optimization selections, corresponding to resonance ($m_X$=260 GeV) search.}
\label{fig:SigOpt:mH260_m_l1jj.pdf}
\end{figure}

\begin{table}[h]
\begin{center}
\tiny\scalebox{0.8}{
\begin{tabular}{c|cccc|cc|c}
\hline
\hline
                          &promptSS  &$V+\gamma$  &QmisID  &Fakes  &Total bkg  &Observed  &signal  \\
\hline
$\Delta R_{min}(l_{2},j)$ &102.50$\pm$4.47(stat.)$\pm$30.75(syst.)  &33.03$\pm$5.51(stat.)$\pm$16.52(syst.)  &35.61$\pm$0.33(stat.)$\pm$11.75(syst.)  &128.89$\pm$8.34(stat.)$\pm$81.32(syst.)  &300.04$\pm$10.96(stat.)$\mp$36.83(syst1.)$\pm$81.32(syst2.)  &343  &0.80$\pm$0.03\\
\hline
$\Delta R_{min}(l_{1},j)$ &49.48$\pm$3.15(stat.)$\pm$14.84(syst.)  &22.72$\pm$5.05(stat.)$\pm$11.36(syst.)  &22.09$\pm$0.21(stat.)$\pm$7.29(syst.)  &92.54$\pm$7.07(stat.)$\pm$58.38(syst.)  &186.83$\pm$9.25(stat.)$\mp$20.06(syst1.)$\pm$58.38(syst2.)  &194  &0.74$\pm$0.03\\
\hline
$M(\ell\ell)$ &18.49$\pm$1.92(stat.)$\pm$5.55(syst.)  &5.31$\pm$2.21(stat.)$\pm$2.65(syst.)  &7.27$\pm$0.08(stat.)$\pm$2.40(syst.)  &36.86$\pm$4.46(stat.)$\pm$23.25(syst.)  &67.93$\pm$5.34(stat.)$\mp$6.60(syst1.)$\pm$23.25(syst2.)  &90  &0.61$\pm$0.03\\
\hline
$M(l_{1}jj)$ &18.15$\pm$1.91(stat.)$\pm$5.44(syst.)  &5.31$\pm$2.21(stat.)$\pm$2.65(syst.)  &7.22$\pm$0.08(stat.)$\pm$2.38(syst.)  &36.34$\pm$4.43(stat.)$\pm$22.93(syst.)  &67.02$\pm$5.31(stat.)$\mp$6.51(syst1.)$\pm$22.93(syst2.)  &89  &0.61$\pm$0.03\\
\hline
\hline
\end{tabular}}
\end{center}

\caption{$m_X$=300~GeV $ee$类别的优化结果。}
%\caption{The unblinded results of $m_X$=300~GeV search in $ee$ channel. }
\label{cutflow_ee_mX300}
\end{table}

\begin{table}[h]
\begin{center}
\tiny\scalebox{0.8}{
\begin{tabular}{c|cccc|cc|c}
\hline
\hline
                          &promptSS  &$V+\gamma$  &QmisID  &Fakes  &Total bkg  &Observed  &signal  \\
\hline
$\Delta R_{min}(l_{2},j)$ &169.68$\pm$6.15(stat.)$\pm$50.90(syst.)  &0.01$\pm$0.01(stat.)$\pm$0.00(syst.)  &0.00$\pm$0.00(stat.)$\pm$0.00(syst.)  &142.15$\pm$8.46(stat.)$\pm$91.75(syst.)  &311.83$\pm$10.46(stat.)$\mp$50.90(syst1.)$\pm$91.75(syst2.)  &245  &1.80$\pm$0.06\\
\hline
$\Delta R_{min}(l_{1},j)$ &97.25$\pm$4.96(stat.)$\pm$29.17(syst.)  &0.01$\pm$0.01(stat.)$\pm$0.00(syst.)  &0.00$\pm$0.00(stat.)$\pm$0.00(syst.)  &120.11$\pm$7.77(stat.)$\pm$77.53(syst.)  &217.36$\pm$9.22(stat.)$\mp$29.17(syst1.)$\pm$77.53(syst2.)  &141  &1.66$\pm$0.06\\
\hline
$M(\ell\ell)$ &50.83$\pm$3.33(stat.)$\pm$15.25(syst.)  &0.00$\pm$0.00(stat.)$\pm$0.00(syst.)  &0.00$\pm$0.00(stat.)$\pm$0.00(syst.)  &78.24$\pm$6.28(stat.)$\pm$50.51(syst.)  &129.07$\pm$7.11(stat.)$\mp$15.25(syst1.)$\pm$50.51(syst2.)  &79  &1.47$\pm$0.05\\
\hline
$M(l_{1}jj)$ &47.98$\pm$3.29(stat.)$\pm$14.39(syst.)  &0.00$\pm$0.00(stat.)$\pm$0.00(syst.)  &0.00$\pm$0.00(stat.)$\pm$0.00(syst.)  &77.49$\pm$6.25(stat.)$\pm$50.02(syst.)  &125.47$\pm$7.06(stat.)$\mp$14.39(syst1.)$\pm$50.02(syst2.)  &74  &1.46$\pm$0.05\\
\hline
\hline
\end{tabular}}
\end{center}

\caption{$m_X$=300~GeV $\mu\mu$类别的优化结果。}
%\caption{The unblinded results of $m_X$=300~GeV search in $\mu\mu$ channel. }
\label{cutflow_mumu_mX300}
\end{table}

\begin{table}[h]
\begin{center}
\tiny\scalebox{0.8}{
\begin{tabular}{c|cccc|cc|c}
\hline
\hline
                          &promptSS  &$V+\gamma$  &QmisID  &Fakes  &Total bkg  &Observed  &signal  \\
\hline
$\Delta R_{min}(l_{2},j)$ &285.01$\pm$7.13(stat.)$\pm$85.50(syst.)  &36.28$\pm$5.39(stat.)$\pm$18.14(syst.)  &9.46$\pm$0.16(stat.)$\pm$3.12(syst.)  &170.21$\pm$9.47(stat.)$\pm$79.34(syst.)  &500.96$\pm$13.02(stat.)$\mp$87.46(syst1.)$\pm$79.34(syst2.)  &596  &2.64$\pm$0.07\\
\hline
$\Delta R_{min}(l_{1},j)$ &182.33$\pm$5.88(stat.)$\pm$54.70(syst.)  &26.12$\pm$4.79(stat.)$\pm$13.06(syst.)  &3.99$\pm$0.07(stat.)$\pm$1.32(syst.)  &139.61$\pm$8.57(stat.)$\pm$64.37(syst.)  &352.05$\pm$11.44(stat.)$\mp$56.25(syst1.)$\pm$64.37(syst2.)  &397  &2.57$\pm$0.07\\
\hline
$M(\ell\ell)$ &68.67$\pm$3.59(stat.)$\pm$20.60(syst.)  &10.73$\pm$2.76(stat.)$\pm$5.36(syst.)  &1.41$\pm$0.03(stat.)$\pm$0.47(syst.)  &66.72$\pm$5.92(stat.)$\pm$30.48(syst.)  &147.52$\pm$7.45(stat.)$\mp$21.29(syst1.)$\pm$30.48(syst2.)  &163  &2.05$\pm$0.06\\
\hline
$M(l_{1}jj)$ &59.41$\pm$3.38(stat.)$\pm$17.82(syst.)  &8.94$\pm$2.50(stat.)$\pm$4.47(syst.)  &1.24$\pm$0.03(stat.)$\pm$0.41(syst.)  &62.74$\pm$5.74(stat.)$\pm$28.94(syst.)  &132.33$\pm$7.12(stat.)$\mp$18.38(syst1.)$\pm$28.94(syst2.)  &144  &1.99$\pm$0.06\\
\hline
\hline
\end{tabular}}
\end{center}

\caption{$m_X$=300~GeV $e\mu$类别的优化结果。}
%\caption{The unblinded results of $m_X$=300~GeV search in $e\mu$ channel. }
\label{cutflow_emu_mX300}
\end{table}

\begin{figure}[h]
\begin{minipage}[t]{0.33\linewidth}
 \centering
 \includegraphics[width=0.9\textwidth,angle=-90]{fig/SigOpt/mH300_m_all_ee.pdf}
 \end{minipage}
 \begin{minipage}[t]{0.33\linewidth}
 \centering
 \includegraphics[width=0.9\textwidth,angle=-90]{fig/SigOpt/mH300_m_all_mumu.pdf}
 \end{minipage}
 \begin{minipage}[t]{0.33\linewidth}
 \centering
 \includegraphics[width=0.9\textwidth,angle=-90]{fig/SigOpt/mH300_m_all_emu.pdf}
 \end{minipage}
 \caption{$m_X$=300 GeV质量点经过优化筛选之后$M(all)$分布。}
%\caption{The unblinded $M(all)$ distribution after all optimization selections, corresponding to resonance ($m_X$=300 GeV) search.}
\label{fig:SigOpt:mH300_m_l1jj.pdf}
\end{figure}

\begin{table}[h]
\begin{center}
\tiny\scalebox{0.65}{
\begin{tabular}{c|cccc|cc|c}
\hline
\hline
                           &promptSS  &$V+\gamma$  &QmisID  &Fakes  &Total bkg  &Observed  &signal  \\  
\hline
$\Delta R_{min}(l_{2},j)$ &52.12$\pm$3.04(stat.)$\pm$15.64(syst.)  &17.85$\pm$4.08(stat.)$\pm$8.93(syst.)  &17.54$\pm$0.26(stat.)$\pm$5.79(syst.)  &69.22$\pm$6.15(stat.)$\pm$47.89(syst.)  &156.73$\pm$7.99(stat.)$\mp$18.91(syst1.)$\pm$47.89(syst2.)  &182  &3.33$\pm$0.08\\
\hline
$\Delta R_{min}(l_{1},j)$ &20.98$\pm$1.96(stat.)$\pm$6.29(syst.)  &3.51$\pm$1.10(stat.)$\pm$1.76(syst.)  &6.68$\pm$0.15(stat.)$\pm$2.20(syst.)  &27.96$\pm$3.91(stat.)$\pm$19.35(syst.)  &59.14$\pm$4.52(stat.)$\mp$6.90(syst1.)$\pm$19.35(syst2.)  &75  &2.34$\pm$0.07\\
\hline
$M(\ell\ell)$ &16.15$\pm$1.83(stat.)$\pm$4.85(syst.)  &1.25$\pm$0.42(stat.)$\pm$0.63(syst.)  &4.50$\pm$0.12(stat.)$\pm$1.49(syst.)  &21.82$\pm$3.46(stat.)$\pm$15.09(syst.)  &43.73$\pm$3.93(stat.)$\mp$5.11(syst1.)$\pm$15.09(syst2.)  &59  &2.27$\pm$0.07\\
\hline
$M(l_{1}jj)$ &11.56$\pm$1.25(stat.)$\pm$3.47(syst.)  &0.83$\pm$0.34(stat.)$\pm$0.41(syst.)  &3.46$\pm$0.10(stat.)$\pm$1.14(syst.)  &19.09$\pm$3.23(stat.)$\pm$13.21(syst.)  &34.94$\pm$3.48(stat.)$\mp$3.67(syst1.)$\pm$13.21(syst2.)  &46  &2.16$\pm$0.07\\
\hline
\hline
\end{tabular}}
\end{center}

\caption{$m_X$=400~GeV $ee$类别的优化结果。}
%\caption{The unblinded results of $m_X$=400~GeV search in $ee$ channel. }
\label{cutflow_ee_mX400}
\end{table}

\begin{table}[h]
\begin{center}
\tiny\scalebox{0.8}{
\begin{tabular}{c|cccc|cc|c}
\hline
\hline
                           &promptSS  &$V+\gamma$  &QmisID  &Fakes  &Total bkg  &Observed  &signal  \\  
\hline
$\Delta R_{min}(l_{2},j)$ &59.36$\pm$3.01(stat.)$\pm$17.81(syst.)  &0.01$\pm$0.01(stat.)$\pm$0.00(syst.)  &0.00$\pm$0.00(stat.)$\pm$0.00(syst.)  &51.25$\pm$4.83(stat.)$\pm$36.79(syst.)  &110.61$\pm$5.69(stat.)$\mp$17.81(syst1.)$\pm$36.79(syst2.)  &99  &1.72$\pm$0.06\\
\hline
$\Delta R_{min}(l_{1},j)$ &25.36$\pm$1.92(stat.)$\pm$7.61(syst.)  &0.00$\pm$0.00(stat.)$\pm$0.00(syst.)  &0.00$\pm$0.00(stat.)$\pm$0.00(syst.)  &18.80$\pm$2.92(stat.)$\pm$13.50(syst.)  &44.17$\pm$3.50(stat.)$\mp$7.61(syst1.)$\pm$13.50(syst2.)  &37  &1.38$\pm$0.05\\
\hline
$M(\ell\ell)$ &18.50$\pm$1.72(stat.)$\pm$5.55(syst.)  &0.00$\pm$0.00(stat.)$\pm$0.00(syst.)  &0.00$\pm$0.00(stat.)$\pm$0.00(syst.)  &17.02$\pm$2.78(stat.)$\pm$12.22(syst.)  &35.51$\pm$3.27(stat.)$\mp$5.55(syst1.)$\pm$12.22(syst2.)  &26  &1.34$\pm$0.05\\
\hline
$M(l_{1}jj)$ &13.82$\pm$1.38(stat.)$\pm$4.15(syst.)  &0.00$\pm$0.00(stat.)$\pm$0.00(syst.)  &0.00$\pm$0.00(stat.)$\pm$0.00(syst.)  &14.09$\pm$2.53(stat.)$\pm$10.11(syst.)  &27.91$\pm$2.88(stat.)$\mp$4.15(syst1.)$\pm$10.11(syst2.)  &19  &1.30$\pm$0.05\\
\hline
\hline
\end{tabular}}
\end{center}

\caption{$m_X$=400~GeV $\mu\mu$类别的优化结果。}
%\caption{The unblinded results of $m_X$=400~GeV search in $\mu\mu$ channel. }
\label{cutflow_mumu_mX400}
\end{table}

\begin{table}[h]
\begin{center}
\tiny\scalebox{0.8}{
\begin{tabular}{c|cccc|cc|c}
\hline
\hline
                           &promptSS  &$V+\gamma$  &QmisID  &Fakes  &Total bkg  &Observed  &signal  \\  
\hline
$\Delta R_{min}(l_{2},j)$ &145.72$\pm$5.10(stat.)$\pm$43.72(syst.)  &23.17$\pm$4.97(stat.)$\pm$11.59(syst.)  &5.06$\pm$0.13(stat.)$\pm$1.67(syst.)  &72.77$\pm$6.07(stat.)$\pm$36.36(syst.)  &246.72$\pm$9.35(stat.)$\mp$45.26(syst1.)$\pm$36.36(syst2.)  &283  &3.33$\pm$0.08 \\
\hline
$\Delta R_{min}(l_{1},j)$ &46.01$\pm$2.73(stat.)$\pm$13.80(syst.)  &7.96$\pm$3.23(stat.)$\pm$3.98(syst.)  &1.69$\pm$0.07(stat.)$\pm$0.56(syst.)  &27.03$\pm$3.75(stat.)$\pm$14.31(syst.)  &82.70$\pm$5.65(stat.)$\mp$14.38(syst1.)$\pm$14.31(syst2.)  &93  &2.34$\pm$0.07 \\
\hline
$M(\ell\ell)$ &33.90$\pm$2.37(stat.)$\pm$10.17(syst.)  &6.54$\pm$3.20(stat.)$\pm$3.27(syst.)  &0.86$\pm$0.04(stat.)$\pm$0.28(syst.)  &20.86$\pm$3.29(stat.)$\pm$11.11(syst.)  &62.16$\pm$5.17(stat.)$\mp$10.69(syst1.)$\pm$11.11(syst2.)  &69  &2.27$\pm$0.07\\
\hline
$M(l_{1}jj)$ &24.13$\pm$1.97(stat.)$\pm$7.24(syst.)  &2.47$\pm$1.25(stat.)$\pm$1.23(syst.)  &0.60$\pm$0.03(stat.)$\pm$0.20(syst.)  &17.82$\pm$3.06(stat.)$\pm$9.80(syst.)  &45.02$\pm$3.84(stat.)$\mp$7.35(syst1.)$\pm$9.80(syst2.)  &57  &2.16$\pm$0.07\\
\hline
\hline
\end{tabular}}
\end{center}

\caption{$m_X$=400~GeV $e\mu$类别的优化结果。}
%\caption{The unblinded results of $m_X$=400~GeV search in $e\mu$ channel. }
\label{cutflow_emu_mX400}
\end{table}

\begin{figure}[h]
\begin{minipage}[t]{0.33\linewidth}
 \centering
 \includegraphics[width=0.9\textwidth,angle=-90]{fig/SigOpt/mH400_m_l1jj_ee.pdf}
 \end{minipage}
 \begin{minipage}[t]{0.33\linewidth}
 \centering
 \includegraphics[width=0.9\textwidth,angle=-90]{fig/SigOpt/mH400_m_l1jj_mumu.pdf}
 \end{minipage}
 \begin{minipage}[t]{0.33\linewidth}
 \centering
 \includegraphics[width=0.9\textwidth,angle=-90]{fig/SigOpt/mH400_m_l1jj_emu.pdf}
 \end{minipage}
 \caption{$m_X$=400 GeV质量点经过优化筛选之后$M(\ell_{1}jj)$分布。}
%\caption{The unblinded $M(\ell_{1}jj)$ distribution after all optimization selections, corresponding to resonance ($m_X$=400 GeV) search.}
\label{fig:SigOpt:mH400_m_l1jj.pdf}
\end{figure}

\begin{table}[h]
\begin{center}
\tiny\scalebox{0.8}{
\begin{tabular}{c|cccc|cc|c}
\hline
\hline
                           &promptSS  &$V+\gamma$  &QmisID  &Fakes  &Total bkg  &Observed  &signal  \\  
\hline
$\Delta R_{min}(l_{2},j)$ &36.10$\pm$2.66(stat.)$\pm$10.83(syst.)  &12.21$\pm$3.77(stat.)$\pm$6.10(syst.)  &11.32$\pm$0.21(stat.)$\pm$3.73(syst.)  &49.72$\pm$5.22(stat.)$\pm$34.39(syst.)  &109.34$\pm$6.96(stat.)$\mp$12.98(syst1.)$\pm$34.39(syst2.)  &117  &1.40$\pm$0.05\\
\hline
$\Delta R_{min}(l_{1},j)$ &13.11$\pm$1.68(stat.)$\pm$3.93(syst.)  &1.89$\pm$0.95(stat.)$\pm$0.94(syst.)  &4.00$\pm$0.12(stat.)$\pm$1.32(syst.)  &17.81$\pm$3.12(stat.)$\pm$12.32(syst.)  &36.81$\pm$3.67(stat.)$\mp$4.26(syst1.)$\pm$12.32(syst2.)  &47  &1.13$\pm$0.04\\
\hline
$M(\ell\ell)$ &5.40$\pm$0.79(stat.)$\pm$1.62(syst.)  &0.36$\pm$0.19(stat.)$\pm$0.18(syst.)  &1.59$\pm$0.08(stat.)$\pm$0.53(syst.)  &6.65$\pm$1.91(stat.)$\pm$4.60(syst.)  &14.01$\pm$2.08(stat.)$\mp$1.71(syst1.)$\pm$4.60(syst2.)  &21  &0.90$\pm$0.04\\
\hline
$M(l_{1}jj)$ &3.92$\pm$0.70(stat.)$\pm$1.17(syst.)  &0.12$\pm$0.05(stat.)$\pm$0.06(syst.)  &1.24$\pm$0.07(stat.)$\pm$0.41(syst.)  &4.03$\pm$1.48(stat.)$\pm$2.79(syst.)  &9.31$\pm$1.64(stat.)$\mp$1.25(syst1.)$\pm$2.79(syst2.)  &14  &0.85$\pm$0.04\\
\hline
\hline
\end{tabular}}
\end{center}

\caption{$m_X$=500~GeV $ee$类别的优化结果。}
%\caption{The unblinded results of $m_X$=500~GeV search in $ee$ channel. }
\label{cutflow_ee_mX500}
\end{table}

\input{tables/cutflow_SigOpt_mumu_mH500.tex}
\caption{$m_X$=500~GeV $\mu\mu$类别的优化结果。}
%\caption{The unblinded results of $m_X$=500~GeV search in $\mu\mu$ channel. }
\label{cutflow_mumu_mX500}
\end{table}

\begin{table}[h]
\begin{center}
\tiny\scalebox{0.65}{
\begin{tabular}{c|cccc|cc|c}
\hline
\hline
                           &promptSS  &$V+\gamma$  &QmisID  &Fakes  &Total bkg  &Observed  &signal  \\  
\hline
$\Delta R_{min}(l_{2},j)$ &71.26$\pm$3.23(stat.)$\pm$21.38(syst.)  &12.92$\pm$3.94(stat.)$\pm$6.46(syst.)  &2.74$\pm$0.10(stat.)$\pm$0.90(syst.)  &40.57$\pm$4.57(stat.)$\pm$20.90(syst.)  &127.48$\pm$6.85(stat.)$\mp$22.35(syst1.)$\pm$20.90(syst2.)  &152  &3.41$\pm$0.07\\
\hline
$\Delta R_{min}(l_{1},j)$ &15.07$\pm$1.61(stat.)$\pm$4.52(syst.)  &0.63$\pm$0.23(stat.)$\pm$0.31(syst.)  &0.53$\pm$0.05(stat.)$\pm$0.18(syst.)  &6.42$\pm$1.86(stat.)$\pm$4.17(syst.)  &22.64$\pm$2.47(stat.)$\mp$4.53(syst1.)$\pm$4.17(syst2.)  &30  &2.19$\pm$0.06\\
\hline
$M(\ell\ell)$ &8.61$\pm$1.15(stat.)$\pm$2.58(syst.)  &0.03$\pm$0.03(stat.)$\pm$0.02(syst.)  &0.27$\pm$0.03(stat.)$\pm$0.09(syst.)  &2.10$\pm$1.04(stat.)$\pm$1.10(syst.)  &11.01$\pm$1.55(stat.)$\mp$2.59(syst1.)$\pm$1.10(syst2.)  &13  &1.94$\pm$0.05\\
\hline
$M(l_{1}jj)$ &7.07$\pm$1.04(stat.)$\pm$2.12(syst.)  &0.03$\pm$0.03(stat.)$\pm$0.01(syst.)  &0.21$\pm$0.02(stat.)$\pm$0.07(syst.)  &1.70$\pm$0.93(stat.)$\pm$0.86(syst.)  &9.01$\pm$1.40(stat.)$\mp$2.12(syst1.)$\pm$0.86(syst2.)  &10  &1.91$\pm$0.05\\
\hline
\hline
\end{tabular}}
\end{center}

\caption{$m_X$=500~GeV $e\mu$类别的优化结果。}
%\caption{The unblinded results of $m_X$=500~GeV search in $e\mu$ channel. }
\label{cutflow_emu_mX500}
\end{table}

\begin{figure}[h]
\begin{minipage}[t]{0.33\linewidth}
 \centering
 \includegraphics[width=0.9\textwidth,angle=-90]{fig/SigOpt/mH500_m_l1jj_ee.pdf}
 \end{minipage}
 \begin{minipage}[t]{0.33\linewidth}
 \centering
 \includegraphics[width=0.9\textwidth,angle=-90]{fig/SigOpt/mH500_m_l1jj_mumu.pdf}
 \end{minipage}
 \begin{minipage}[t]{0.33\linewidth}
 \centering
 \includegraphics[width=0.9\textwidth,angle=-90]{fig/SigOpt/mH500_m_l1jj_emu.pdf}
 \end{minipage}
 \caption{$m_X$=500 GeV质量点经过优化筛选之后$M(\ell_{1}jj)$分布。}
%\caption{The unblinded $M(\ell_{1}jj)$ distribution after all optimization selections, corresponding to resonance ($m_X$=500 GeV) search.}
\label{fig:SigOpt:mH500_m_l1jj.pdf}
\end{figure}

\clearpage
\subsubsection{$SS$搜寻筛选结果}
\input{tables/cutflow_SigOpt_ee_HSS_lowmass.tex}
\caption{$m_X$=280 GeV, $m_X$=300 GeV and $m_X$=320 GeV(fixing $m_S$=135 GeV) $ee$类别的优化结果。}
%\caption{The unblinded results of the searches for $m_X$=280 GeV, $m_X$=300 GeV and $m_X$=320 GeV(fixing $m_S$=135 GeV) in $ee$ channel.}
\label{cutflow_ee_HSS_lowmass}
\end{table}

\begin{table}[h]
\begin{center}
\tiny\scalebox{0.8}{
\begin{tabular}{c|ccccc}
\hline
\hline
                           &promptSS  &$V+\gamma$  &QmisID  &Fakes  &Total bkg    \\  
\hline
$\Delta R_{min}(l_{2},j)$ &229.53$\pm$6.81(stat.)$\pm$68.86(syst.)  &0.01$\pm$0.01(stat.)$\pm$0.00(syst.)  &0.00$\pm$0.00(stat.)$\pm$0.00(syst.)  &188.45$\pm$9.74(stat.)$\pm$121.65(syst.)  &417.99$\pm$11.88(stat.)$\mp$68.86(syst1.)$\pm$121.65(syst2.) \\
\hline
$\Delta R_{min}(l_{1},j)$ &137.67$\pm$5.13(stat.)$\pm$41.30(syst.)  &0.01$\pm$0.01(stat.)$\pm$0.00(syst.)  &0.00$\pm$0.00(stat.)$\pm$0.00(syst.)  &121.81$\pm$7.83(stat.)$\pm$78.63(syst.)  &259.49$\pm$9.36(stat.)$\mp$41.30(syst1.)$\pm$78.63(syst2.)  \\
\hline
$M(\ell\ell)$ &65.75$\pm$3.73(stat.)$\pm$19.73(syst.)  &0.00$\pm$0.00(stat.)$\pm$0.00(syst.)  &0.00$\pm$0.00(stat.)$\pm$0.00(syst.)  &87.44$\pm$6.63(stat.)$\pm$56.44(syst.)  &153.19$\pm$7.61(stat.)$\mp$19.73(syst1.)$\pm$56.44(syst2.) \\
\hline
$M(l_{1}jj)$ &21.62$\pm$2.05(stat.)$\pm$6.49(syst.)  &0.00$\pm$0.00(stat.)$\pm$0.00(syst.)  &0.00$\pm$0.00(stat.)$\pm$0.00(syst.)  &43.90$\pm$4.70(stat.)$\pm$28.34(syst.)  &65.52$\pm$5.13(stat.)$\mp$6.49(syst1.)$\pm$28.34(syst2.)  \\
\hline
                          &$X280, S135$  &$X300, S135$  &$X320, S135$ &Observed \\
\hline
$\Delta R_{min}(l_{2},j)$  &6.37$\pm$0.27 &7.13$\pm$0.29  &7.74$\pm$0.31 &355\\
$\Delta R_{min}(l_{1},j)$  &5.71$\pm$0.26 &6.23$\pm$0.28  &6.82$\pm$0.29 &210\\
$M(\ell\ell)$              &5.51$\pm$0.26 &5.51$\pm$0.26  &5.43$\pm$0.26 &104\\
$M(l_{1}jj)$               &4.43$\pm$0.24 &4.31$\pm$0.22  &3.96$\pm$0.23 &46\\
\hline
\hline
\end{tabular}}
\end{center}

\caption{$m_X$=280 GeV, $m_X$=300 GeV and $m_X$=320 GeV(fixing $m_S$=135 GeV) $\mu\mu$类别的优化结果。}
%\caption{The unblinded results of the searches for $m_X$=280 GeV, $m_X$=300 GeV and $m_X$=320 GeV(fixing $m_S$=135 GeV) in $\mu\mu$ channel.}
\label{cutflow_mumu_HSS_lowmass}
\end{table}

\input{tables/cutflow_SigOpt_emu_HSS_lowmass.tex}
\caption{$m_X$=280 GeV, $m_X$=300 GeV and $m_X$=320 GeV(fixing $m_S$=135 GeV) $e\mu$类别的优化结果。}
%\caption{The unblinded results of the searches for $m_X$=280 GeV, $m_X$=300 GeV and $m_X$=320 GeV(fixing $m_S$=135 GeV) in $e\mu$ channel.}
\label{cutflow_emu_HSS_lowmass}
\end{table}

\begin{figure}[h]
\begin{minipage}[t]{0.33\linewidth}
 \centering
 \includegraphics[width=0.9\textwidth,angle=-90]{fig/SigOpt/H300_S135_m_l1jj_ee.pdf}
 \end{minipage}
 \begin{minipage}[t]{0.33\linewidth}
 \centering
 \includegraphics[width=0.9\textwidth,angle=-90]{fig/SigOpt/H300_S135_m_l1jj_mumu.pdf}
 \end{minipage}
 \begin{minipage}[t]{0.33\linewidth}
 \centering
 \includegraphics[width=0.9\textwidth,angle=-90]{fig/SigOpt/H300_S135_m_l1jj_emu.pdf}
 \end{minipage}
 \caption{经过$SS$低质量点优化筛选条件的$M(\ell_{1}jj)$分布,图中展示的信号是$m_X$=300 GeV, $m_S$=135 GeV。}
%\caption{The unblinded $M(\ell_{1}jj)$ distribution after all optimization selections, corresponding to $SS$ low mass search. The signal shown here is $m_X$=300 GeV, $m_S$=135 GeV.}
\label{fig:SigOpt:H300_S135_m_l1jj.pdf}
\end{figure}

\input{tables/cutflow_SigOpt_ee_HSS_highmass.tex}
\caption{$m_S$=135 GeV, $m_S$=145 GeV, $m_S$=155 GeV and $m_S$=165 GeV(fixing $m_X$=340 GeV) $ee$类别的优化结果。}
%\caption{The unblinded results of the searches for $m_S$=135 GeV, $m_S$=145 GeV, $m_S$=155 GeV and $m_S$=165 GeV(fixing $m_X$=340 GeV) in $ee$ channel.}
\label{cutflow_ee_HSS_highmass}
\end{table}

\begin{table}[h]
\begin{center}
\tiny\scalebox{0.8}{
\begin{tabular}{c|ccccc}
\hline
\hline
                           &promptSS  &$V+\gamma$  &QmisID  &Fakes  &Total bkg    \\  
\hline
$\Delta R_{min}(l_{2},j)$ &116.18$\pm$4.57(stat.)$\pm$34.85(syst.)  &0.01$\pm$0.01(stat.)$\pm$0.00(syst.)  &0.00$\pm$0.00(stat.)$\pm$0.00(syst.)  &102.27$\pm$6.82(stat.)$\pm$73.42(syst.)  &218.45$\pm$8.21(stat.)$\mp$34.85(syst1.)$\pm$73.42(syst2.)  \\
\hline
$\Delta R_{min}(l_{1},j)$ &72.61$\pm$3.46(stat.)$\pm$21.78(syst.)  &0.01$\pm$0.01(stat.)$\pm$0.00(syst.)  &0.00$\pm$0.00(stat.)$\pm$0.00(syst.)  &64.78$\pm$5.43(stat.)$\pm$46.50(syst.)  &137.39$\pm$6.44(stat.)$\mp$21.78(syst1.)$\pm$46.50(syst2.)  \\
\hline
$M(\ell\ell)$ &39.34$\pm$2.56(stat.)$\pm$11.80(syst.)  &0.01$\pm$0.01(stat.)$\pm$0.00(syst.)  &0.00$\pm$0.00(stat.)$\pm$0.00(syst.)  &55.61$\pm$5.03(stat.)$\pm$39.92(syst.)  &94.96$\pm$5.64(stat.)$\mp$11.80(syst1.)$\pm$39.92(syst2.)  \\
\hline
$M(l_{1}jj)$ &22.92$\pm$1.79(stat.)$\pm$6.88(syst.)  &0.00$\pm$0.00(stat.)$\pm$0.00(syst.)  &0.00$\pm$0.00(stat.)$\pm$0.00(syst.)  &39.33$\pm$4.23(stat.)$\pm$28.24(syst.)  &62.25$\pm$4.59(stat.)$\mp$6.88(syst1.)$\pm$28.24(syst2.)  \\
\hline
                             &X340, S135  &X340, S145  &X340, S155  &X340, S165 &Observed\\
\hline
$\Delta R_{min}(l_{2},j)$  &5.92$\pm$0.26 &19.02$\pm$0.83  &38.96$\pm$1.50 &64.54$\pm$2.19 &172\\
$\Delta R_{min}(l_{1},j)$  &5.39$\pm$0.25 &17.52$\pm$0.81  &36.36$\pm$1.47 &61.47$\pm$2.14 &113\\
$M(\ell\ell)$              &4.89$\pm$0.24 &16.51$\pm$0.79  &35.05$\pm$1.44 &61.31$\pm$2.14 &66\\
$M(l_{1}jj)$               &4.34$\pm$0.23 &14.90$\pm$0.75  &32.26$\pm$1.39 &56.55$\pm$2.06 &38\\
\hline
\hline
\end{tabular}}
\end{center}

\caption{$m_S$=135 GeV, $m_S$=145 GeV, $m_S$=155 GeV and $m_S$=165 GeV(fixing $m_X$=340 GeV) $\mu\mu$类别的优化结果。}
%\caption{The unblinded results of the searches for $m_S$=135 GeV, $m_S$=145 GeV, $m_S$=155 GeV and $m_S$=165 GeV(fixing $m_X$=340 GeV) in $\mu\mu$ channel.}
\label{cutflow_mumu_HSS_highmass}
\end{table}

\input{tables/cutflow_SigOpt_emu_HSS_highmass.tex}
\caption{$m_S$=135 GeV, $m_S$=145 GeV, $m_S$=155 GeV and $m_S$=165 GeV(fixing $m_X$=340 GeV) $e\mu$类别的优化结果。}
%\caption{The unblinded results of the searches for $m_S$=135 GeV, $m_S$=145 GeV, $m_S$=155 GeV and $m_S$=165 GeV(fixing $m_X$=340 GeV) in $e\mu$ channel.}
\label{cutflow_emu_HSS_highmass}
\end{table}

\begin{figure}[h]
\begin{minipage}[t]{0.33\linewidth}
 \centering
 \includegraphics[width=0.9\textwidth,angle=-90]{fig/SigOpt/H340_S145_m_l1jj_ee.pdf}
 \end{minipage}
 \begin{minipage}[t]{0.33\linewidth}
 \centering
 \includegraphics[width=0.9\textwidth,angle=-90]{fig/SigOpt/H340_S145_m_l1jj_mumu.pdf}
 \end{minipage}
 \begin{minipage}[t]{0.33\linewidth}
 \centering
 \includegraphics[width=0.9\textwidth,angle=-90]{fig/SigOpt/H340_S145_m_l1jj_emu.pdf}
 \end{minipage}
  \caption{经过$SS$高质量点优化筛选条件的$M(\ell_{1}jj)$分布,图中展示的信号是$m_X$=340 GeV, $m_S$=145 GeV。}
%\caption{The unblinded $M(\ell_{1}jj)$ distribution after all optimization selections, corresponding to $SS$ high mass search. The signal shown here is $m_X$=340 GeV, $m_S$=145 GeV.}
\label{fig:SigOpt:H340_S165_m_l1jj.pdf}
\end{figure}
