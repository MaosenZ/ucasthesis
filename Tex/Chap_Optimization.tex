\chapter{信号优化}\label{chap:signal_optimization}
在初步筛选之后,为了进一步加强信号显著性,利用动力学性质进行信号优化是有必要的。因为每个信号质量点的动力学性质差别
较大,所以,每个质量点都会进行信号优化过程,而后通过各自
动力学选择条件之后的区域才作为每个质量点的最终信号区。

\section{优化策略}
MVA方法用于确定不同运动学变量的分离能力,并考虑所有变量之间的相关性。最终,前五个运动学变量用于形成优化选择,分别是$M(\ell\ell)$,$\Delta R_{min}(\ell_{2}, j)$,$\Delta R_{min}(\ell_{1}, j)$ ,$ M_{\ell_{1} jj}$和$M(all)$,它们具有很强的分离能力,而且相互之间的相关性很低(图~\ref{fig:correlation_check})。通常,$M(\ell\ell)$和$ M_{\ell_{1} jj} $对低质量点敏感,而其余对高质量和非共振信号敏感。
基于这些知识,$\Delta R_{min}(\ell_{1}, j)$,$M(\ell\ell)$,$ M_{\ell_{1}jj}$和$M(all)$用于在低质量搜索中形成优化削减,而$\Delta R_{min}(\ell_{2}, j)$,$\Delta R_{min}(\ell_{1}, j)$,$M(\ell\ell)$和$M_{\ell_{1}jj}$用于高质量搜索中。它们的相应分布分别见图~\ref{fig:SigOpt_low_kine}和图~\ref{fig:SigOpt_high_kine}。


\section{信号效率}


\section{优化结果}
