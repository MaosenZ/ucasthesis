%---------------------------------------------------------------------------%
%->> 封面信息及生成
%---------------------------------------------------------------------------%
%-
%-> 中文封面信息
%-
\confidential{}% 密级:只有涉密论文才填写
\schoollogo{scale=0.095}{ucas_logo}% 校徽
\title{基于ATLAS实验寻找$t\bar{t}h$和$hh$产生} % 论文中文题目
\author{周茂森}% 论文作者
\advisor{方亚泉~研究员~中国科学院高能物理研究所}% 指导教师:姓名 专业技术职务 工作单位
%\advisors{}% 指导老师附加信息 或 第二指导老师信息
\degree{博士}% 学位:学士、硕士、博士
\degreetype{理学}% 学位类别:理学、工学、工程、医学等
\major{粒子物理与原子核物理}% 二级学科专业名称
\institute{中国科学院高能物理研究所}% 院系名称
\date{2019~年~6~月}% 毕业日期:夏季为6月、冬季为12月
%-
%-> 英文封面信息
%-
\TITLE{Search for $t\bar{t}h$ and $hh$ productions with the ATLAS detector} % 论文英文题目
\AUTHOR{Zhou Maosen}% 论文作者
\ADVISOR{Supervisor: Professor Fang Yaquan}% 指导教师
\DEGREE{Doctor}% 学位:Bachelor, Master, Doctor。封面格式将根据英文学位名称自动切换,请确保拼写准确无误
\DEGREETYPE{Natural Science}% 学位类别:Philosophy, Natural Science, Engineering, Economics, Agriculture 等
\THESISTYPE{thesis}% 论文类型:thesis, dissertation
\MAJOR{Particle Physics and Nuclear Physics}% 二级学科专业名称
\INSTITUTE{Institute of High Energy Physics, Chinese Academy of Sciences}% 院系名称
\DATE{June, 2019}% 毕业日期:夏季为June、冬季为December
%-
%-> 生成封面
%-
\maketitle% 生成中文封面
\MAKETITLE% 生成英文封面
%-
%-> 作者声明
%-
\makedeclaration% 生成声明页
%-
%-> 中文摘要
%-
\chapter*{摘\quad 要}\chaptermark{摘\quad 要}% 摘要标题
\setcounter{page}{1}% 开始页码
\pagenumbering{Roman}% 页码符号

%本文是中国科学院大学学位论文模板ucasthesis的使用说明文档。主要内容为介绍\LaTeX{}文档类ucasthesis的用法,以及如何使用\LaTeX{}快速高效地撰写学位论文。
本文论述通过多轻子衰变道寻找标准模型(SM)希格斯粒子关联顶夸克对产生模式,使用2015年到2017年ATLAS探测器收集的积分亮度为$\int \mathcal{L}dt~=~80~\text{fb}^{-1}$的质子-质子对撞数据。
本文主要涉及末态是由一个轻子和两个强子化衰变的$\tau$组成的分析道\ltwotau ,该分析道最终给出信号强度为$\mu_{t\bar{t}h}$=$\sigma/\sigma_{\text{SM}}$=$1.00^{+1.22}_{-1.02}$,相当于
$0.96\sigma$的信号显著性。

本文也论述通过多轻子道寻找一对衰变到四个$W$玻色子的中性标量玻色子,使用2015年到2016年ATLAS探测器收集的积分亮度为$\int \mathcal{L}dt=36.1~\text{fb}^{-1}$的质子-质子对撞数据。
该搜寻包括三种信号模型:SM非共振态产生($hh$),共振态产生($X\rightarrow hh$)和有两个重质量类希格斯粒子的共振态产生($X\rightarrow SS$)。通过三个衰变道研究:本文重点关注的相同电荷双轻子(2lss),三轻子(3l)以及四轻子(4l)。联合三个衰变道结果,没有发现明显超出SM迹象。95\%置信度下SM非共振态产生截面观测(期望)上限值为160倍(120)倍预期。
$X\rightarrow hh$在260 GeV $\leq m_X\leq$ 500 GeV的质量区间内的产生截面观测(期望)上限值为9.3 (10) pb到2.8 (2.6) pb。
$X\rightarrow SS$在280 GeV $\leq m_X\leq$ 340 GeV, 135 GeV $\leq m_S\leq$ 165 GeV的质量区间内产生截面观测(期望)上限值为2.5 (2.5) pb到0.16 (0.17) pb。

\keywords{ATLAS, ITk, $t\bar{t}h$, \ltwotau, $hh$, $4W$}% 中文关键词
%-
%-> 英文摘要
%-
\chapter*{Abstract}\chaptermark{Abstract}% 摘要标题

%This paper is a help documentation for the \LaTeX{} class ucasthesis, which is  a thesis template for the University of Chinese Academy of Sciences. The main content is about how to use the ucasthesis, as well as how to write thesis efficiently by using \LaTeX{}.
%The search for the Standard Model (SM) Higgs boson production in association with a top quark pair ($t\bar{t}h$) allows a direct measurement of the Higgs coupling to the heaviest fermion and can constrain effects of new physics beyond the SM in the top coupling sector. 
This thesis presents a search for the Standard Model (SM) Higgs boson production in association with a top quark pair ($t\bar{t}h$)
 in an inclusive multileptonic final state with the full 2015, 2016 and 2017 proton-proton collision dataset collected by
the ATLAS detector at center of mass energy of 13 TeV, corresponding to an integrated luminosity of $\int \mathcal{L}dt=80~\text{fb}^{-1}$.
The sub-channel being focused on is \ltwotau with the signature of one light lepton and two hadronic taus. A best-fit value for the strength of the $t\bar{t}h$ production cross section
with respect to the SM prediction of $\mu_{t\bar{t}h}=1.00^{+1.22}_{-1.02}$ is expected, corresponding the significance of 0.96$\sigma$.

A search for a pair of neutral, scalar bosons decaying to four $W$-bosons in multilepton final states is also performed using 36.1 fb$^{-1}$ dataset collected in 2015 and 2016 with the ATLAS detector. This search uses three production models: non-resonant production, resonant production, and resonant production of a pair of heavy Higgs-like scalars. Three final states, classified by the number of leptons, are analysed: two same-sign leptons (2lss) that is this thesis's focus, three leptons, and four leptons. No significant excess of events is observed above the background expectation using a combination of the three channels. An observed (expected) 95\% confidence-level upper limit of 160 (120) times the SM prediction of non-resonant Higgs pair production cross-section is set from a combined analysis of the three final states. Upper limits are set on the production cross-section times branching ratio of a heavy scalar $X$
 decaying into a Higgs boson pair in the mass range of 260 GeV $\leq m_X\leq$ 500 GeV and the observed (expected) limits range from 9.3 (10) pb to 2.8 (2.6) pb. Upper limits are set on the production cross-section times branching ratio of a heavy scalar $X$ decaying into a pair of heavy scalars $S$ for mass ranges of 280 GeV $\leq m_X\leq$ 340 GeV and 135 GeV $\leq m_S\leq$ 165 GeV and the observed (expected) limits range from 2.5 (2.5) pb to 0.16 (0.17) pb.

\KEYWORDS{ATLAS, ITk, $t\bar{t}h$, \ltwotau, $hh$, $4W$}% 英文关键词
%---------------------------------------------------------------------------%
